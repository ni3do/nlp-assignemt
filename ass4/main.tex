\documentclass[a4paper,12pt]{ETHexercise}
\usepackage{bbm}

\textheight24cm
\topmargin-2.5cm
\oddsidemargin+0cm
\textwidth16.3cm

\usepackage{amsmath}
\usepackage{amsfonts}
\usepackage{amssymb}
\usepackage{color}
\usepackage[latin1]{inputenc}
\usepackage{ifthen}
\usepackage{enumerate}
\usepackage{lastpage}
\usepackage{graphicx}
%\usepackage{subfigure}
\usepackage{subcaption}
\usepackage{bm}
\usepackage{tikz}
\usepackage{pgfplots}
\usepackage{float}
\usepackage{nicefrac}
\usepackage{epsfig}
\usepackage{etoolbox}
\usepackage{framed}
\usepackage{enumitem}
\usepackage{hyperref}
\usepackage{datetime}
\usepackage{url}
%\usepackage{algorithm}
\usepackage{svg}
\usepackage[ruled,vlined, linesnumbered]{algorithm2e}
\usepackage{setspace}
\usepackage[htt]{hyphenat}
\usepackage{qtree}
\usepackage{forest}
\usepackage{inconsolata} % for texttt{}


\usepackage[noend]{algpseudocode}
\algnewcommand{\parState}[1]{\State%
  \parbox[t]{\dimexpr\linewidth-\algmargin}{\strut\hangindent=\algorithmicindent \hangafter=1 #1\strut}}

\algrenewcommand\algorithmicindent{1.0em}%
\renewcommand\algorithmicdo{:}
\renewcommand\algorithmicthen{:}
%%%%% NEW MATH DEFINITIONS %%%%%

\usepackage{amsmath,amsfonts,bm}
\usepackage{xifthen}

% Added definitions

% Derivatev
\newcommand{\deriv}[3][]{
	\ifthenelse{\isempty{#1}}
	{\frac{\partial #2}{\partial #3}}
	{\frac{\partial^#1 #2}{\partial #3} }
}

% Mark sections of captions for referring to divisions of figures
\newcommand{\figleft}{{\em (Left)}}
\newcommand{\figcenter}{{\em (Center)}}
\newcommand{\figright}{{\em (Right)}}
\newcommand{\figtop}{{\em (Top)}}
\newcommand{\figbottom}{{\em (Bottom)}}
\newcommand{\captiona}{{\em (a)}}
\newcommand{\captionb}{{\em (b)}}
\newcommand{\captionc}{{\em (c)}}
\newcommand{\captiond}{{\em (d)}}

% Highlight a newly defined term
\newcommand{\newterm}[1]{{\bf #1}}


% Figure reference, lower-case.
\def\figref#1{figure~\ref{#1}}
% Figure reference, capital. For start of sentence
\def\Figref#1{Figure~\ref{#1}}
\def\twofigref#1#2{figures \ref{#1} and \ref{#2}}
\def\quadfigref#1#2#3#4{figures \ref{#1}, \ref{#2}, \ref{#3} and \ref{#4}}
% Section reference, lower-case.
\def\secref#1{section~\ref{#1}}
% Section reference, capital.
\def\Secref#1{Section~\ref{#1}}
% Reference to two sections.
\def\twosecrefs#1#2{sections \ref{#1} and \ref{#2}}
% Reference to three sections.
\def\secrefs#1#2#3{sections \ref{#1}, \ref{#2} and \ref{#3}}
% Reference to an equation, lower-case.
\def\eqref#1{equation~\ref{#1}}
% Reference to an equation, upper case
\def\Eqref#1{Equation~\ref{#1}}
% A raw reference to an equation---avoid using if possible
\def\plaineqref#1{\ref{#1}}
% Reference to a chapter, lower-case.
\def\chapref#1{chapter~\ref{#1}}
% Reference to an equation, upper case.
\def\Chapref#1{Chapter~\ref{#1}}
% Reference to a range of chapters
\def\rangechapref#1#2{chapters\ref{#1}--\ref{#2}}
% Reference to an algorithm, lower-case.
\def\algref#1{algorithm~\ref{#1}}
% Reference to an algorithm, upper case.
\def\Algref#1{Algorithm~\ref{#1}}
\def\twoalgref#1#2{algorithms \ref{#1} and \ref{#2}}
\def\Twoalgref#1#2{Algorithms \ref{#1} and \ref{#2}}
% Reference to a part, lower case
\def\partref#1{part~\ref{#1}}
% Reference to a part, upper case
\def\Partref#1{Part~\ref{#1}}
\def\twopartref#1#2{parts \ref{#1} and \ref{#2}}

\def\ceil#1{\lceil #1 \rceil}
\def\floor#1{\lfloor #1 \rfloor}
\def\1{\bm{1}}
\newcommand{\train}{\mathcal{D}}
\newcommand{\valid}{\mathcal{D_{\mathrm{valid}}}}
\newcommand{\test}{\mathcal{D_{\mathrm{test}}}}

\def\eps{{\epsilon}}


% Random variables
\def\reta{{\textnormal{$\eta$}}}
\def\ra{{\textnormal{a}}}
\def\rb{{\textnormal{b}}}
\def\rc{{\textnormal{c}}}
\def\rd{{\textnormal{d}}}
\def\re{{\textnormal{e}}}
\def\rf{{\textnormal{f}}}
\def\rg{{\textnormal{g}}}
\def\rh{{\textnormal{h}}}
\def\ri{{\textnormal{i}}}
\def\rj{{\textnormal{j}}}
\def\rk{{\textnormal{k}}}
\def\rl{{\textnormal{l}}}
% rm is already a command, just don't name any random variables m
\def\rn{{\textnormal{n}}}
\def\ro{{\textnormal{o}}}
\def\rp{{\textnormal{p}}}
\def\rq{{\textnormal{q}}}
\def\rr{{\textnormal{r}}}
\def\rs{{\textnormal{s}}}
\def\rt{{\textnormal{t}}}
\def\ru{{\textnormal{u}}}
\def\rv{{\textnormal{v}}}
\def\rw{{\textnormal{w}}}
\def\rx{{\textnormal{x}}}
\def\ry{{\textnormal{y}}}
\def\rz{{\textnormal{z}}}

% Random vectors
\def\rvepsilon{{\mathbf{\epsilon}}}
\def\rvtheta{{\mathbf{\theta}}}
\def\rva{{\mathbf{a}}}
\def\rvb{{\mathbf{b}}}
\def\rvc{{\mathbf{c}}}
\def\rvd{{\mathbf{d}}}
\def\rve{{\mathbf{e}}}
\def\rvf{{\mathbf{f}}}
\def\rvg{{\mathbf{g}}}
\def\rvh{{\mathbf{h}}}
\def\rvu{{\mathbf{i}}}
\def\rvj{{\mathbf{j}}}
\def\rvk{{\mathbf{k}}}
\def\rvl{{\mathbf{l}}}
\def\rvm{{\mathbf{m}}}
\def\rvn{{\mathbf{n}}}
\def\rvo{{\mathbf{o}}}
\def\rvp{{\mathbf{p}}}
\def\rvq{{\mathbf{q}}}
\def\rvr{{\mathbf{r}}}
\def\rvs{{\mathbf{s}}}
\def\rvt{{\mathbf{t}}}
\def\rvu{{\mathbf{u}}}
\def\rvv{{\mathbf{v}}}
\def\rvw{{\mathbf{w}}}
\def\rvx{{\mathbf{x}}}
\def\rvy{{\mathbf{y}}}
\def\rvz{{\mathbf{z}}}

% Elements of random vectors
\def\erva{{\textnormal{a}}}
\def\ervb{{\textnormal{b}}}
\def\ervc{{\textnormal{c}}}
\def\ervd{{\textnormal{d}}}
\def\erve{{\textnormal{e}}}
\def\ervf{{\textnormal{f}}}
\def\ervg{{\textnormal{g}}}
\def\ervh{{\textnormal{h}}}
\def\ervi{{\textnormal{i}}}
\def\ervj{{\textnormal{j}}}
\def\ervk{{\textnormal{k}}}
\def\ervl{{\textnormal{l}}}
\def\ervm{{\textnormal{m}}}
\def\ervn{{\textnormal{n}}}
\def\ervo{{\textnormal{o}}}
\def\ervp{{\textnormal{p}}}
\def\ervq{{\textnormal{q}}}
\def\ervr{{\textnormal{r}}}
\def\ervs{{\textnormal{s}}}
\def\ervt{{\textnormal{t}}}
\def\ervu{{\textnormal{u}}}
\def\ervv{{\textnormal{v}}}
\def\ervw{{\textnormal{w}}}
\def\ervx{{\textnormal{x}}}
\def\ervy{{\textnormal{y}}}
\def\ervz{{\textnormal{z}}}

% Random matrices
\def\rmA{{\mathbf{A}}}
\def\rmB{{\mathbf{B}}}
\def\rmC{{\mathbf{C}}}
\def\rmD{{\mathbf{D}}}
\def\rmE{{\mathbf{E}}}
\def\rmF{{\mathbf{F}}}
\def\rmG{{\mathbf{G}}}
\def\rmH{{\mathbf{H}}}
\def\rmI{{\mathbf{I}}}
\def\rmJ{{\mathbf{J}}}
\def\rmK{{\mathbf{K}}}
\def\rmL{{\mathbf{L}}}
\def\rmM{{\mathbf{M}}}
\def\rmN{{\mathbf{N}}}
\def\rmO{{\mathbf{O}}}
\def\rmP{{\mathbf{P}}}
\def\rmQ{{\mathbf{Q}}}
\def\rmR{{\mathbf{R}}}
\def\rmS{{\mathbf{S}}}
\def\rmT{{\mathbf{T}}}
\def\rmU{{\mathbf{U}}}
\def\rmV{{\mathbf{V}}}
\def\rmW{{\mathbf{W}}}
\def\rmX{{\mathbf{X}}}
\def\rmY{{\mathbf{Y}}}
\def\rmZ{{\mathbf{Z}}}

% Elements of random matrices
\def\ermA{{\textnormal{A}}}
\def\ermB{{\textnormal{B}}}
\def\ermC{{\textnormal{C}}}
\def\ermD{{\textnormal{D}}}
\def\ermE{{\textnormal{E}}}
\def\ermF{{\textnormal{F}}}
\def\ermG{{\textnormal{G}}}
\def\ermH{{\textnormal{H}}}
\def\ermI{{\textnormal{I}}}
\def\ermJ{{\textnormal{J}}}
\def\ermK{{\textnormal{K}}}
\def\ermL{{\textnormal{L}}}
\def\ermM{{\textnormal{M}}}
\def\ermN{{\textnormal{N}}}
\def\ermO{{\textnormal{O}}}
\def\ermP{{\textnormal{P}}}
\def\ermQ{{\textnormal{Q}}}
\def\ermR{{\textnormal{R}}}
\def\ermS{{\textnormal{S}}}
\def\ermT{{\textnormal{T}}}
\def\ermU{{\textnormal{U}}}
\def\ermV{{\textnormal{V}}}
\def\ermW{{\textnormal{W}}}
\def\ermX{{\textnormal{X}}}
\def\ermY{{\textnormal{Y}}}
\def\ermZ{{\textnormal{Z}}}

% Vectors
\def\vzero{{\bm{0}}}
\def\vone{{\bm{1}}}
\def\vmu{{\bm{\mu}}}
\def\vtheta{{\bm{\theta}}}
\def\va{{\bm{a}}}
\def\vb{{\bm{b}}}
\def\vc{{\bm{c}}}
\def\vd{{\bm{d}}}
\def\ve{{\bm{e}}}
\def\vf{{\bm{f}}}
\def\vg{{\bm{g}}}
\def\vh{{\bm{h}}}
\def\vi{{\bm{i}}}
\def\vj{{\bm{j}}}
\def\vk{{\bm{k}}}
\def\vl{{\bm{l}}}
\def\vm{{\bm{m}}}
\def\vn{{\bm{n}}}
\def\vo{{\bm{o}}}
\def\vp{{\bm{p}}}
\def\vq{{\bm{q}}}
\def\vr{{\bm{r}}}
\def\vs{{\bm{s}}}
\def\vt{{\bm{t}}}
\def\vu{{\bm{u}}}
\def\vv{{\bm{v}}}
\def\vw{{\bm{w}}}
\def\vx{{\bm{x}}}
\def\vy{{\bm{y}}}
\def\vz{{\bm{z}}}

% Elements of vectors
\def\evalpha{{\alpha}}
\def\evbeta{{\beta}}
\def\evepsilon{{\epsilon}}
\def\evlambda{{\lambda}}
\def\evomega{{\omega}}
\def\evmu{{\mu}}
\def\evpsi{{\psi}}
\def\evsigma{{\sigma}}
\def\evtheta{{\theta}}
\def\eva{{a}}
\def\evb{{b}}
\def\evc{{c}}
\def\evd{{d}}
\def\eve{{e}}
\def\evf{{f}}
\def\evg{{g}}
\def\evh{{h}}
\def\evi{{i}}
\def\evj{{j}}
\def\evk{{k}}
\def\evl{{l}}
\def\evm{{m}}
\def\evn{{n}}
\def\evo{{o}}
\def\evp{{p}}
\def\evq{{q}}
\def\evr{{r}}
\def\evs{{s}}
\def\evt{{t}}
\def\evu{{u}}
\def\evv{{v}}
\def\evw{{w}}
\def\evx{{x}}
\def\evy{{y}}
\def\evz{{z}}

% Matrix
\def\mA{{\bm{A}}}
\def\mB{{\bm{B}}}
\def\mC{{\bm{C}}}
\def\mD{{\bm{D}}}
\def\mE{{\bm{E}}}
\def\mF{{\bm{F}}}
\def\mG{{\bm{G}}}
\def\mH{{\bm{H}}}
\def\mI{{\bm{I}}}
\def\mJ{{\bm{J}}}
\def\mK{{\bm{K}}}
\def\mL{{\bm{L}}}
\def\mM{{\bm{M}}}
\def\mN{{\bm{N}}}
\def\mO{{\bm{O}}}
\def\mP{{\bm{P}}}
\def\mQ{{\bm{Q}}}
\def\mR{{\bm{R}}}
\def\mS{{\bm{S}}}
\def\mT{{\bm{T}}}
\def\mU{{\bm{U}}}
\def\mV{{\bm{V}}}
\def\mW{{\bm{W}}}
\def\mX{{\bm{X}}}
\def\mY{{\bm{Y}}}
\def\mZ{{\bm{Z}}}
\def\mBeta{{\bm{\beta}}}
\def\mPhi{{\bm{\Phi}}}
\def\mLambda{{\bm{\Lambda}}}
\def\mSigma{{\bm{\Sigma}}}

% Tensor
\DeclareMathAlphabet{\mathsfit}{\encodingdefault}{\sfdefault}{m}{sl}
\SetMathAlphabet{\mathsfit}{bold}{\encodingdefault}{\sfdefault}{bx}{n}
\newcommand{\tens}[1]{\bm{\mathsfit{#1}}}
\def\tA{{\tens{A}}}
\def\tB{{\tens{B}}}
\def\tC{{\tens{C}}}
\def\tD{{\tens{D}}}
\def\tE{{\tens{E}}}
\def\tF{{\tens{F}}}
\def\tG{{\tens{G}}}
\def\tH{{\tens{H}}}
\def\tI{{\tens{I}}}
\def\tJ{{\tens{J}}}
\def\tK{{\tens{K}}}
\def\tL{{\tens{L}}}
\def\tM{{\tens{M}}}
\def\tN{{\tens{N}}}
\def\tO{{\tens{O}}}
\def\tP{{\tens{P}}}
\def\tQ{{\tens{Q}}}
\def\tR{{\tens{R}}}
\def\tS{{\tens{S}}}
\def\tT{{\tens{T}}}
\def\tU{{\tens{U}}}
\def\tV{{\tens{V}}}
\def\tW{{\tens{W}}}
\def\tX{{\tens{X}}}
\def\tY{{\tens{Y}}}
\def\tZ{{\tens{Z}}}


% Graph
\def\gA{{\mathcal{A}}}
\def\gB{{\mathcal{B}}}
\def\gC{{\mathcal{C}}}
\def\gD{{\mathcal{D}}}
\def\gE{{\mathcal{E}}}
\def\gF{{\mathcal{F}}}
\def\gG{{\mathcal{G}}}
\def\gH{{\mathcal{H}}}
\def\gI{{\mathcal{I}}}
\def\gJ{{\mathcal{J}}}
\def\gK{{\mathcal{K}}}
\def\gL{{\mathcal{L}}}
\def\gM{{\mathcal{M}}}
\def\gN{{\mathcal{N}}}
\def\gO{{\mathcal{O}}}
\def\gP{{\mathcal{P}}}
\def\gQ{{\mathcal{Q}}}
\def\gR{{\mathcal{R}}}
\def\gS{{\mathcal{S}}}
\def\gT{{\mathcal{T}}}
\def\gU{{\mathcal{U}}}
\def\gV{{\mathcal{V}}}
\def\gW{{\mathcal{W}}}
\def\gX{{\mathcal{X}}}
\def\gY{{\mathcal{Y}}}
\def\gZ{{\mathcal{Z}}}

% Sets
\def\sA{{\mathbb{A}}}
\def\sB{{\mathbb{B}}}
\def\sC{{\mathbb{C}}}
\def\sD{{\mathbb{D}}}
% Don't use a set called E, because this would be the same as our symbol
% for expectation.
\def\sF{{\mathbb{F}}}
\def\sG{{\mathbb{G}}}
\def\sH{{\mathbb{H}}}
\def\sI{{\mathbb{I}}}
\def\sJ{{\mathbb{J}}}
\def\sK{{\mathbb{K}}}
\def\sL{{\mathbb{L}}}
\def\sM{{\mathbb{M}}}
\def\sN{{\mathbb{N}}}
\def\sO{{\mathbb{O}}}
\def\sP{{\mathbb{P}}}
\def\sQ{{\mathbb{Q}}}
\def\sR{{\mathbb{R}}}
\def\sS{{\mathbb{S}}}
\def\sT{{\mathbb{T}}}
\def\sU{{\mathbb{U}}}
\def\sV{{\mathbb{V}}}
\def\sW{{\mathbb{W}}}
\def\sX{{\mathbb{X}}}
\def\sY{{\mathbb{Y}}}
\def\sZ{{\mathbb{Z}}}

% Entries of a matrix
\def\emLambda{{\Lambda}}
\def\emA{{A}}
\def\emB{{B}}
\def\emC{{C}}
\def\emD{{D}}
\def\emE{{E}}
\def\emF{{F}}
\def\emG{{G}}
\def\emH{{H}}
\def\emI{{I}}
\def\emJ{{J}}
\def\emK{{K}}
\def\emL{{L}}
\def\emM{{M}}
\def\emN{{N}}
\def\emO{{O}}
\def\emP{{P}}
\def\emQ{{Q}}
\def\emR{{R}}
\def\emS{{S}}
\def\emT{{T}}
\def\emU{{U}}
\def\emV{{V}}
\def\emW{{W}}
\def\emX{{X}}
\def\emY{{Y}}
\def\emZ{{Z}}
\def\emSigma{{\Sigma}}

% entries of a tensor
% Same font as tensor, without \bm wrapper
\newcommand{\etens}[1]{\mathsfit{#1}}
\def\etLambda{{\etens{\Lambda}}}
\def\etA{{\etens{A}}}
\def\etB{{\etens{B}}}
\def\etC{{\etens{C}}}
\def\etD{{\etens{D}}}
\def\etE{{\etens{E}}}
\def\etF{{\etens{F}}}
\def\etG{{\etens{G}}}
\def\etH{{\etens{H}}}
\def\etI{{\etens{I}}}
\def\etJ{{\etens{J}}}
\def\etK{{\etens{K}}}
\def\etL{{\etens{L}}}
\def\etM{{\etens{M}}}
\def\etN{{\etens{N}}}
\def\etO{{\etens{O}}}
\def\etP{{\etens{P}}}
\def\etQ{{\etens{Q}}}
\def\etR{{\etens{R}}}
\def\etS{{\etens{S}}}
\def\etT{{\etens{T}}}
\def\etU{{\etens{U}}}
\def\etV{{\etens{V}}}
\def\etW{{\etens{W}}}
\def\etX{{\etens{X}}}
\def\etY{{\etens{Y}}}
\def\etZ{{\etens{Z}}}

% The true underlying data generating distribution
\newcommand{\pdata}{p_{\rm{data}}}
% The empirical distribution defined by the training set
\newcommand{\ptrain}{\hat{p}_{\rm{data}}}
\newcommand{\Ptrain}{\hat{P}_{\rm{data}}}
% The model distribution
\newcommand{\pmodel}{p_{\rm{model}}}
\newcommand{\Pmodel}{P_{\rm{model}}}
\newcommand{\ptildemodel}{\tilde{p}_{\rm{model}}}
% Stochastic autoencoder distributions
\newcommand{\pencode}{p_{\rm{encoder}}}
\newcommand{\pdecode}{p_{\rm{decoder}}}
\newcommand{\precons}{p_{\rm{reconstruct}}}

\newcommand{\laplace}{\mathrm{Laplace}} % Laplace distribution

\newcommand{\E}{\mathbb{E}}
\newcommand{\Ls}{\mathcal{L}}
\newcommand{\R}{\mathbb{R}}
\newcommand{\emp}{\tilde{p}}
\newcommand{\lr}{\alpha}
\newcommand{\reg}{\lambda}
\newcommand{\rect}{\mathrm{rectifier}}
\newcommand{\softmax}{\mathrm{softmax}}
\newcommand{\sigmoid}{\sigma}
\newcommand{\softplus}{\zeta}
\newcommand{\KL}{D_{\mathrm{KL}}}
\newcommand{\Var}{\mathrm{Var}}
\newcommand{\standarderror}{\mathrm{SE}}
\newcommand{\Cov}{\mathrm{Cov}}
% Wolfram Mathworld says $L^2$ is for function spaces and $\ell^2$ is for vectors
% But then they seem to use $L^2$ for vectors throughout the site, and so does
% wikipedia.
\newcommand{\normlzero}{L^0}
\newcommand{\normlone}{L^1}
\newcommand{\normltwo}{L^2}
\newcommand{\normlp}{L^p}
\newcommand{\normmax}{L^\infty}

\newcommand{\parents}{Pa} % See usage in notation.tex. Chosen to match Daphne's book.

\DeclareMathOperator*{\argmax}{arg\,max}
\DeclareMathOperator*{\argmin}{arg\,min}

\DeclareMathOperator{\sign}{sign}
\DeclareMathOperator{\Tr}{Tr}
\let\ab\allowbreak

%\newcommand{\linetofill}{\ \\\hphantom{\hspace{0mm}}\hrulefill\ \\}
\newcommand{\linetofill}{\ \\\hphantom{\hspace{0mm}}\dotfill\ \\}
\newcommand{\vlinetofill}[1]{\\\rule{#1 mm}{0.1mm}\ \\}
\newcommand{\cbox}{
  \setlength{\unitlength}{1pt}
  \begin{picture}(10,10)
    \put(0,0){\line(1,0){8}} \put(0,8){\line(1,0){8}}
    \put(0,0){\line(0,1){8}} \put(8,0){\line(0,1){8}}
  \end{picture}
}
\newcommand{\bigcbox}{
  \setlength{\unitlength}{3pt}
  \begin{picture}(10,10)(1,1)
    \put(0,0){\line(1,0){8}} \put(0,8){\line(1,0){8}}
    \put(0,0){\line(0,1){8}} \put(8,0){\line(0,1){8}}
  \end{picture}
}
% points and boxes on the right
\newcommand{\points}[1]{
  \ \\[-5mm]
  \hphantom{\ }\hfill\textbf{#1 pts \bigcbox \hspace{-6mm}}
  \vspace{5mm}
}
% multiple choice checkboxes
%\newcommand{\boxt}{\hspace*{2em} $[~~]$ \textsf{True}}
%\newcommand{\boxf}{\hspace*{2em} $[~~]$ \textsf{False}}
\newcommand{\boxt}{\hspace*{2em} $\square$ \textsf{True}}
\newcommand{\boxf}{\hspace*{2em} $\square$ \textsf{False}}
\newcommand{\justif}{\textsf{Justification}:  \rule{0ex}{2em}\dotfill\\\rule{0ex}{2em}\dotfill}
\newcommand{\checkboxWithJustification}{\boxt \boxf\\ \justif\\}
\newcommand{\checkbox}{\boxt \boxf}
\usepackage{tikz}
\usepackage{xcolor}
\usepackage{amssymb}
\usepackage{amsmath}
\usepackage{amsthm}
\usepackage{pgfplots}
\pgfmathdeclarefunction{gauss}{2}{%
  \pgfmathparse{1/(#2*sqrt(2*pi))*exp(-((x-#1)^2)/(2*#2^2))}%
}
\renewcommand{\S}{{\cal S}}
\usepackage{pgfplots}
\pgfplotsset{compat=newest}
\pgfplotsset{small, every non boxed x axis/.append style={x axis line style=-},
  every non boxed y axis/.append style={y axis line style=-}}

\newcommand{\timestamp}{\ddmmyyyydate\today \,\,- \currenttime h}
%%% Local Variables:
%%% mode: latex
%%% TeX-master: "exam"
%%% End:

% Numbers
\usepackage[group-separator={,}]{siunitx}

\usepackage{cleveref}
\crefname{section}{\S}{\S\S}
\Crefname{section}{\S}{\S\S}
\crefname{table}{Tab.}{}
\crefname{figure}{Fig.}{}
\crefname{algorithm}{Algorithm}{}
\crefname{equation}{eq.}{}
\crefname{appendix}{App.}{}
\crefname{thm}{Theorem}{}
\crefname{prop}{Proposition}{}
\crefname{cor}{Corollary}{}
\crefname{observation}{Observation}{}
\crefname{assumption}{Assumption}{}
\crefformat{section}{\S#2#1#3}

\usepackage{multirow}
\title{NLP Assignment}
\begin{document}
\setserie{1}


\newcommand{\pair}[2]{{\langle #1 , #2 \rangle}}
\newcommand{\score}[2]{\text{score}_{\theta}(\langle #1, #2 \rangle, \boldsymbol{w})}
\newcommand{\sscore}[1]{\text{score}_{\theta}(#1, \boldsymbol{w})}
\newcommand*{\eos}{\text{EOS}}

\lectureheader{Prof. Ryan Cotterell}
{}
{\Large Natural Language Processing}{Fall 2022}
\begin{center}
	{\Huge Simon Wachter: Assignment 4}\\
	\quad\newline
	siwachte@ethz.ch, 19-920-198\\
	\quad\newline
	\timestamp
\end{center}
\begin{question}\\
	\begin{subquestion}
		\begin{align}
			\sum_{w \in \Sigma^*} \tilde{p}(w) & = \sum_{w \in \Sigma^*, |w|=1} \tilde{p}(w) + \sum_{w \in \Sigma^*, |w|=2} \tilde{p}(w) + \sum_{w \in \Sigma^*, |w|=3} \tilde{p}(w) + \dots \\
			                                   & = \sum_{w \in \Sigma} \tilde{p}(w) + \sum_{w \in \Sigma^*, |w|=2} \tilde{p}(w) + \sum_{w \in \Sigma^*, |w|=3} \tilde{p}(w) + \dots          \\
			                                   & = 1 + \sum_{w \in \Sigma^*, |w|=2} \tilde{p}(w) + \sum_{w \in \Sigma^*, |w|=3} \tilde{p}(w) + \dots                                         \\
			                                   & = 1 + \sum_{w \in \Sigma, |w|=1} \sum_{w' \in \Sigma^*} \tilde{p}(w) \tilde{p}(w') + \sum_{w \in \Sigma^*, |w|=3} \tilde{p}(w) + \dots      \\
			                                   & = 1 + \sum_{w \in \Sigma} \tilde{p}(w) \sum_{w' \in \Sigma^*, |w|=1} \tilde{p}(w') + \sum_{w \in \Sigma^*, |w|=3} \tilde{p}(w) + \dots      \\
			                                   & = 1 + \sum_{w \in \Sigma} \tilde{p}(w) \sum_{w' \in \Sigma} \tilde{p}(w') + \sum_{w \in \Sigma^*, |w|=3} \tilde{p}(w) + \dots               \\
			                                   & = 1 + 1 \cdot 1 + \sum_{w \in \Sigma^*, |w|=3} \tilde{p}(w) + \dots                                                                         \\
			                                   & = 1 + 1 \cdot 1 + \sum_{w \in \Sigma}  \sum_{w \in \Sigma^*, |w|=2} \tilde{p}(w) \tilde{p}(w') + \dots                                      \\
			                                   & = 1 + 1 \cdot 1 + \sum_{w \in \Sigma} \tilde{p}(w) \sum_{w \in \Sigma^*, |w|=2} \tilde{p}(w') + \dots                                       \\
			                                   & = 1 + 1 \cdot 1 + 1 \cdot 1 \cdot 1 + \dots                                                                                                 \\
			                                   & = \sum_{i=1}^{\infty} 1^{i}                                                                                         \label{1a_inf_series}
		\end{align}
		We can see that the series in \eqref{1a_inf_series} diverges to $\infty$.
	\end{subquestion}
	\begin{subquestion}
		We first state some equations that are used later:
		\begin{align}
			\sum_{w \in \Sigma \cup \{\eos\}} p(w) & = 1 \label{1b_sum_p}                                                                                 \\
			\Rightarrow \sum_{w \in \Sigma} p(w)   & = 1 - p(\eos)                                                                                        \\
			\sum_{w \in \Sigma} p(w)               & < 1                                                                                                  \\
			\sum_{w \in \Sigma} p(\eos)p(w)        & = p(\eos) \sum_{w \in \Sigma} p(w)                                                                   \\
			\sum_{w \in \Sigma^*, |w|=0} p(w)      & = p(\eos)                          &  & \text{because we have only one EOS symbol} \label{1b_sum_p0} \\
		\end{align}
		We define $l$ as:
		\begin{align}
			l & = \sum_{w \in \Sigma} p(w) \label{1b_def_l} \\
			  & = (1 - p(\eos)) \label{1b_def_l2}           \\
			l & < 1
		\end{align}
		Now we can formulate the given equation:
		\begin{align}
			\sum_{w \in \Sigma^*} p(w) & =  \sum_{w \in \Sigma^*, |w|=0} p(w) + \sum_{w \in \Sigma^*, |w|=1} p(w) + \sum_{w \in \Sigma^*, |w|=2} p(w) + \sum_{w \in \Sigma^*, |w|=3} p(w) + \dots                                                       \\
			                           & = p(\eos) + \sum_{w \in \Sigma^*, |w|=1} p(w) + \sum_{w \in \Sigma^*, |w|=2} p (w) + \sum_{w \in \Sigma^*, |w|=3} p(w) + \dots                                                                                 \\
			                           & = p(\eos) \left( 1 + \sum_{w \in \Sigma^*, |w|=1} \tilde{p}(w) + \sum_{w \in \Sigma^*, |w|=2} \tilde{p}(w) + \sum_{w \in \Sigma^*, |w|=3} \tilde{p}(w) + \dots                                         \right) \\
			                           & = p(\eos) \left( 1 + \sum_{w \in \Sigma^*, |w|=1} \tilde{p}(w) + \sum_{w \in \Sigma^*, |w|=2}  \tilde{p}(w) + \sum_{w \in \Sigma^*, |w|=3}  \tilde{p}(w) + \dots  \right)                                      \\
			                           & =  p(\eos) \left( 1 + \sum_{w \in \Sigma} p(w) + \sum_{w \in \Sigma^*, |w|=2}  \tilde{p}(w) + \sum_{w \in \Sigma^*, |w|=3}  \tilde{p}(w) + \dots \right)                                                       \\
			                           & = p(\eos) \left( 1 + l + \sum_{w \in \Sigma^*, |w|=2}  \tilde{p}(w) + \sum_{w \in \Sigma^*, |w|=3}  \tilde{p}(w) + \dots  \right)                                                                              \\
			                           & = p(\eos) \left( 1 + l + \sum_{w \in \Sigma} \sum_{w' \in \Sigma} p(w)p(w') + \sum_{w \in \Sigma^*, |w|=3}  \tilde{p}(w) + \dots \right)                                                                       \\
			                           & = p(\eos) \left( 1 + l +  \sum_{w \in \Sigma} p(w) \sum_{w' \in \Sigma} p(w') + \sum_{w \in \Sigma^*, |w|=3}  \tilde{p}(w) + \dots  \right)                                                                    \\
			                           & = p(\eos) \left( 1 + l + l^2 + \sum_{w \in \Sigma^*, |w|=3}  \tilde{p}(w) + \dots   \right)                                                                                                                    \\
			                           & = p(\eos) \left( 1 + l + l^2 + l^3 + \dots     \right)                                                                                                                                                         \\
			                           & = p(\eos) \left(\sum_{n=0}^{\infty} l^n                                                 \right)                                                                                                                \\
			                           & = p(\eos) \frac{1}{1-l} \quad\quad \text{(limit geom. series, because $l < 1$)} \label{1b_sum_p2}                                                                                                              \\
			                           & = \frac{p(\eos)}{1 - (1-p(\eos))}                                                                                                                                                                              \\
			                           & = \frac{p(\eos)}{p(\eos)}                                                                                                                                                                                      \\
			                           & = 1
		\end{align}
	\end{subquestion}
	\begin{subquestion}
		\begin{align}
			\sum_{u \in \Sigma^*} p(wu) & = \sum_{u \in \Sigma^*} p(\eos | wu) p_{pre}(u|w)p_{pre}(w)                &  & \text{by def.}    \\
			                            & = p_{pre}(w) \left( \sum_{u \in \Sigma^*} p(\eos | wu) p_{pre}(u|w)\right)                        \\
			                            & = p_{pre}(w) \left( \sum_{u \in \Sigma^*}p(u | w) \right)                  &  & \text{def. p(w)}  \\
			                            & = p_{pre}(w) \left( \sum_{u \in \Sigma^*} \frac{p(w|u)p(u)}{p(w)} \right)  &  & \text{Bayes rule} \\
			                            & = p_{pre}(w) \left( \frac{p(w)}{p(w)}\right)                               &  & \text{Bayes rule} \\
			                            & = p_{pre}(w)
		\end{align}
	\end{subquestion}
	\begin{subquestion}
		We use CKY with the $(+, \times)$-semiring. Further we use $\log p$ insted of $p$ for our score function. These operations will lead to $p(w_1 \dots w_n | S)$ being calculated at the top of the tree because we fill the tree according to:
		\begin{align}
			\text{s} [i,k,X] & = \sum_{k}^{} \sum_{i} \sum_{j} \exp (\text{score} (X \rightarrow YZ)) \times \text{s} [i,j,Y] \times \text{s} [j,k,Z] \\
			                 & = \exp(\log(p(YZ | X))) \times p(w_i \dots w_j | Y) \times p(w_{j+1} \dots w_k | Z)                                    \\
			                 & = p(YZ | X) \times p(w_i \dots w_j | Y) \times p(w_{j+1} \dots w_k | Z)                                                \\
			                 & = p(w_i \dots w_k | X)
		\end{align}
		Where $k$, $i$ and $j$ are the indices that the CKY algorithm goes over.
		We can the easily multiply $p(w_1 \dots w_n | S)$ by $p(\text{EOS})$ and get the desired result.
	\end{subquestion}
	\begin{subquestion}
		\begin{align}
			\sum_{u \in \Sigma^*} p(wu) & = \sum_{u \in \Sigma^*} p_{\text{inside}}(wu | S)     \\
			                            & = p(X \overset{*}{\Rightarrow} w_i \dots w_k)         \\
			                            & = p(S \overset{*}{\Rightarrow} wv) \label{eq:1e_conc} \\
		\end{align}
		\cref*{eq:1e_conc} holds, because summing over all possible suffixes is equivalent to allow all different derivation trees that produce $v$.
	\end{subquestion}
	\begin{subquestion}
		Following the hint given in the exercise, we create a matrix $M \in |\mathcal{N}| \times |\mathcal{N}|$ where each entry $M_{X,Y}$ corresponds to the probability of deriving $Y$ from $X$:
		\begin{align}
			M_{Y,X} & = p(X \rightarrow Y\alpha)
		\end{align}
		We can see that by multiplying in the inside semiring this matrix with itself we get the probability of deriving $i$ from $j$ in two steps and so on. Thus, the kleene star over $M$ will yield the desired property. To derive the kleene star we use Lehmann's algorithm:
		\begin{align}
			M^* & =\bigotimes_{k=0}^{\infty} M      \\
			    & = I + M \bigotimes_{k=0}^{\infty} \\
			    & = I + MM^*
		\end{align}
		\begin{align}
			M^*  - MM^* & =I                                 \\
			(I- M)M^*   & = I                                \\
			M^*         & = (I - M)^{-1} \label{eq:m_kleene}
		\end{align}
		As stated in the lecture notes, for \cref*{eq:m_kleene} to holde we require the largest eigenvalue of $M$ to be smaller than $1$. This translates to the condition that all diagonal entries of $M$ have to be $< 1$. The diagonal entries correspond to the probability of deriving a symbol from itself. This probability has to be smaller than $1$ in our case, since otherwise we would not be able to generate any other symbol from it than itself, which would results in infinite sentences.\\
		Then, using \cref{eq:m_kleene} we can calculate $M^*$ in $\mathcal{O}(|\mathcal{N}|^3)$. The entries in the matrix now correspond to:
		\begin{align}
			M^*_{X,Y} = p[X \overset{*}{\Rightarrow} Y\alpha] & = p_{\text{lc}}(Y | X)
		\end{align}
		From this matrix we further need to derive the $p_{\text{lc}}(YZ | X)$. We iterate over all possible choices for $X, X',Y,Z$ to derive:
		\begin{align}
			p_{\text{lc}}(YZ | X) & = \sum_{X' \in \Sigma} p_{\text{lc}}(X' | X) p(X' \Rightarrow YZ) \label{eq:lc}
		\end{align}
		Which can be done in $\mathcal{O}(|\mathcal{N}|^4)$.
	\end{subquestion}
	\begin{subquestion}
		\begin{align}
			p_{pre}(w_i \dots w_k | X)                                    & = \sum_{j=1}^{k-1} \sum_{Y,Z \in \mathcal{N}} {\color{red} p(X \overset{*}{\Rightarrow} YZ\alpha)}{\color{green}p(Y \overset{*}{\Rightarrow} w_i \dots w_j)}{\color{blue} p(Z \overset{*}{\Rightarrow} w_{j+1} \dots w_k)} \label{eq:pre} \\
			{\color{red}p(X \overset{*}{\Rightarrow} YZ\alpha)}           & = \text{probability of deriving the two subtrees $Y$ and $Z$ from $X$} \label{eq:pre2}                                                                                                                                                    \\
			{\color{green}p(Y \overset{*}{\Rightarrow} w_i \dots w_j)}    & = \text{probability of deriving the string $w_i \dots w_j$ from $Y$} \label{eq:pre3}                                                                                                                                                      \\
			{\color{blue}p(Z \overset{*}{\Rightarrow} w_{j+1} \dots w_k)} & = \text{probability of deriving the string $w_{j+1} \dots w_k$ from $Z$} \label{eq:pre4}                                                                                                                                                  \\
			                                                              & \sum_{j=1}^{k-1} \sum_{Y,Z \in \mathcal{N}} {\color{red} p(X \overset{*}{\Rightarrow} YZ\alpha)}{\color{green}p(Y \overset{*}{\Rightarrow} w_i \dots w_j)}{\color{blue} p(Z \overset{*}{\Rightarrow} w_{j+1} \dots w_k)}                  \\& =\sum_{j=1}^{k-1} \sum_{Y,Z \in \mathcal{N}}p_{\text{lc}}(YZ | X) p_{\text{inside}}(w_i \dots w_j | Y) p_{\text{pre}}(w_{j+1} \dots w_k | Z) \label{eq:pre5}                                                                               \\
		\end{align}
		The three colored parts correspond to the subtrees shown in Figure 1 b) on the exercise sheet.
	\end{subquestion}
	\begin{subquestion}
		We can see that if we have all precomputation done, then the calculation of $p_{pre}(w')$ can be done in $\mathcal{O}(1)$ for a given subsequence $w'$ of $w$. As $|w| = N$, this has runtime $\mathcal{O}(N)$.\\
		Now for the precompution, we first look at $p_{\text{lc}}(YZ | X)$. As shown in exercise f) this can be done in $\mathcal{O}(|\mathcal{N}|^4)$.
		$p_{\text{pre}}(w_k | X)$ can be calculated as follows:
		\begin{align}
			p_{\text{pre}}(w_k | X) & = p_{\text{pre}}(X \overset{*}{\Rightarrow} w_k)                  &  & \text{eq (12) on ex. sheet} \\
			                        & = \sum_{Y \in \mathcal{N}} p_{\text{lc}}(Y|X)p(Y \Rightarrow w_k)
		\end{align}
		Due to our PCFG being in CNF, this can be done in $\mathcal{O}(|\mathcal{N}|)$ and further for all $X \in \mathcal{N}$ in $\mathcal{O}(|\mathcal{N}|^2)$.\\
		We calculate $p_{\text{inside}}$ with the CKY algorithm and save all intermediate values. The size of our grammar is $\mathcal{O}(|\mathcal{N}|^3)$. Thus, CKY runs in $\mathcal{O}(N^3 |\mathcal{N}|^3)$. As in exercise d) we use the $\log p$ as our score function.\\
		We can now calculate:
		\begin{align}
			p_{\text{pre}}(w_{i} \dots w_k | X) & = \sum_{j=i}^{k-1} \sum_{Y,Z \in \mathcal{N}} p_{\text{lc}}(YZ | X) p_{\text{inside}}(w_i \dots w_j | Y) p_{\text{pre}}(w_{j+1} \dots w_k | Z)
		\end{align}
		Where we start with $i = k-1$ and iterate until the desired $i$. For each iteration, the outer sum is in $\mathcal{O}(N)$, the inner sum is in $\mathcal{O}(|\mathcal{N}|^2)$ and the computation is in $\mathcal{O}(1)$. Thus, the total runtime is $\mathcal{O}(N^2 |\mathcal{N}|^3)$ as we have to go over the length of $w$ and over all $X$. Concluding then the runtime of the whole procedure for all prefixes of $w$ is:
		\begin{align}
			\mathcal{O}(N (|\mathcal{N}|^4 + N^3 |\mathcal{N}|^3 + N^2 |\mathcal{N}|^3)) & = \mathcal{O}(N^4 |\mathcal{N}|^3 + N |\mathcal{N}|^4)
		\end{align}
	\end{subquestion}
	\begin{subquestion}
		We now use CKY to calculate the prefix probabilities. The precomputation of the left corner and inside probabilities can be left as is. The CKY update function will be:
		\begin{align}
			\text{chart}[i,k,X] & = \sum_{j=i}^{k-1} \sum_{Y,Z \in \mathcal{N}} p_{\text{lc}}(YZ | X) p_{\text{inside}}(w_i \dots w_j | Y) \text{chart}[j+1, k, Z] \label{eq:cky} \\
			                    & = p_{pre}(w_i \dots w_k | X)
		\end{align}
		Hence, we have the same runtime as the CKY algorithm plus all precomputation steps. To initialize the chart, we have to calculate all $\text{chart}[i,i,X] = p_{pre}(w_i | X) | \forall i$, takes $N$ times $\mathcal{O}(|\mathcal{N}|^2)$, thus $\mathcal{O}(N |\mathcal{N}|^2)$. So the total runtime is:
		\begin{align}
			\mathcal{O}(N |\mathcal{N}|^2 + N^3 |\mathcal{N}|^3 + N^2 |\mathcal{N}|^3 + |\mathcal{N}|^4) & = \mathcal{O}(N^3 |\mathcal{N}|^3 + |\mathcal{N}|^4)
		\end{align}
	\end{subquestion}
\end{question}
\end{document}

%%% Local Variables:
%%% mode: latex
%%% TeX-master: t
%%% End: