\documentclass[a4paper,12pt]{ETHexercise}
\usepackage{bbm}

\textheight24cm
\topmargin-2.5cm
\oddsidemargin+0cm
\textwidth16.3cm

\usepackage{amsmath}
\usepackage{amsfonts}
\usepackage{amssymb}
\usepackage{color}
\usepackage[latin1]{inputenc}
\usepackage{ifthen}
\usepackage{enumerate}
\usepackage{lastpage}
\usepackage{graphicx}
%\usepackage{subfigure}
\usepackage{subcaption}
\usepackage{bm}
\usepackage{tikz}
\usepackage{pgfplots}
\usepackage{float}
\usepackage{nicefrac}
\usepackage{epsfig}
\usepackage{etoolbox}
\usepackage{framed}
\usepackage{enumitem}
\usepackage{hyperref}
\usepackage{datetime}
\usepackage{url}
%\usepackage{algorithm}
\usepackage{svg}
\usepackage[ruled,vlined, linesnumbered]{algorithm2e}
\usepackage{setspace}
\usepackage[htt]{hyphenat}
\usepackage{qtree}
\usepackage{forest}
\usepackage{inconsolata} % for texttt{}


\usepackage[noend]{algpseudocode}
\algnewcommand{\parState}[1]{\State%
  \parbox[t]{\dimexpr\linewidth-\algmargin}{\strut\hangindent=\algorithmicindent \hangafter=1 #1\strut}}

\algrenewcommand\algorithmicindent{1.0em}%
\renewcommand\algorithmicdo{:}
\renewcommand\algorithmicthen{:}
%%%%% NEW MATH DEFINITIONS %%%%%

\usepackage{amsmath,amsfonts,bm}
\usepackage{xifthen}

% Added definitions

% Derivatev
\newcommand{\deriv}[3][]{
	\ifthenelse{\isempty{#1}}
	{\frac{\partial #2}{\partial #3}}
	{\frac{\partial^#1 #2}{\partial #3} }
}

% Mark sections of captions for referring to divisions of figures
\newcommand{\figleft}{{\em (Left)}}
\newcommand{\figcenter}{{\em (Center)}}
\newcommand{\figright}{{\em (Right)}}
\newcommand{\figtop}{{\em (Top)}}
\newcommand{\figbottom}{{\em (Bottom)}}
\newcommand{\captiona}{{\em (a)}}
\newcommand{\captionb}{{\em (b)}}
\newcommand{\captionc}{{\em (c)}}
\newcommand{\captiond}{{\em (d)}}

% Highlight a newly defined term
\newcommand{\newterm}[1]{{\bf #1}}


% Figure reference, lower-case.
\def\figref#1{figure~\ref{#1}}
% Figure reference, capital. For start of sentence
\def\Figref#1{Figure~\ref{#1}}
\def\twofigref#1#2{figures \ref{#1} and \ref{#2}}
\def\quadfigref#1#2#3#4{figures \ref{#1}, \ref{#2}, \ref{#3} and \ref{#4}}
% Section reference, lower-case.
\def\secref#1{section~\ref{#1}}
% Section reference, capital.
\def\Secref#1{Section~\ref{#1}}
% Reference to two sections.
\def\twosecrefs#1#2{sections \ref{#1} and \ref{#2}}
% Reference to three sections.
\def\secrefs#1#2#3{sections \ref{#1}, \ref{#2} and \ref{#3}}
% Reference to an equation, lower-case.
\def\eqref#1{equation~\ref{#1}}
% Reference to an equation, upper case
\def\Eqref#1{Equation~\ref{#1}}
% A raw reference to an equation---avoid using if possible
\def\plaineqref#1{\ref{#1}}
% Reference to a chapter, lower-case.
\def\chapref#1{chapter~\ref{#1}}
% Reference to an equation, upper case.
\def\Chapref#1{Chapter~\ref{#1}}
% Reference to a range of chapters
\def\rangechapref#1#2{chapters\ref{#1}--\ref{#2}}
% Reference to an algorithm, lower-case.
\def\algref#1{algorithm~\ref{#1}}
% Reference to an algorithm, upper case.
\def\Algref#1{Algorithm~\ref{#1}}
\def\twoalgref#1#2{algorithms \ref{#1} and \ref{#2}}
\def\Twoalgref#1#2{Algorithms \ref{#1} and \ref{#2}}
% Reference to a part, lower case
\def\partref#1{part~\ref{#1}}
% Reference to a part, upper case
\def\Partref#1{Part~\ref{#1}}
\def\twopartref#1#2{parts \ref{#1} and \ref{#2}}

\def\ceil#1{\lceil #1 \rceil}
\def\floor#1{\lfloor #1 \rfloor}
\def\1{\bm{1}}
\newcommand{\train}{\mathcal{D}}
\newcommand{\valid}{\mathcal{D_{\mathrm{valid}}}}
\newcommand{\test}{\mathcal{D_{\mathrm{test}}}}

\def\eps{{\epsilon}}


% Random variables
\def\reta{{\textnormal{$\eta$}}}
\def\ra{{\textnormal{a}}}
\def\rb{{\textnormal{b}}}
\def\rc{{\textnormal{c}}}
\def\rd{{\textnormal{d}}}
\def\re{{\textnormal{e}}}
\def\rf{{\textnormal{f}}}
\def\rg{{\textnormal{g}}}
\def\rh{{\textnormal{h}}}
\def\ri{{\textnormal{i}}}
\def\rj{{\textnormal{j}}}
\def\rk{{\textnormal{k}}}
\def\rl{{\textnormal{l}}}
% rm is already a command, just don't name any random variables m
\def\rn{{\textnormal{n}}}
\def\ro{{\textnormal{o}}}
\def\rp{{\textnormal{p}}}
\def\rq{{\textnormal{q}}}
\def\rr{{\textnormal{r}}}
\def\rs{{\textnormal{s}}}
\def\rt{{\textnormal{t}}}
\def\ru{{\textnormal{u}}}
\def\rv{{\textnormal{v}}}
\def\rw{{\textnormal{w}}}
\def\rx{{\textnormal{x}}}
\def\ry{{\textnormal{y}}}
\def\rz{{\textnormal{z}}}

% Random vectors
\def\rvepsilon{{\mathbf{\epsilon}}}
\def\rvtheta{{\mathbf{\theta}}}
\def\rva{{\mathbf{a}}}
\def\rvb{{\mathbf{b}}}
\def\rvc{{\mathbf{c}}}
\def\rvd{{\mathbf{d}}}
\def\rve{{\mathbf{e}}}
\def\rvf{{\mathbf{f}}}
\def\rvg{{\mathbf{g}}}
\def\rvh{{\mathbf{h}}}
\def\rvu{{\mathbf{i}}}
\def\rvj{{\mathbf{j}}}
\def\rvk{{\mathbf{k}}}
\def\rvl{{\mathbf{l}}}
\def\rvm{{\mathbf{m}}}
\def\rvn{{\mathbf{n}}}
\def\rvo{{\mathbf{o}}}
\def\rvp{{\mathbf{p}}}
\def\rvq{{\mathbf{q}}}
\def\rvr{{\mathbf{r}}}
\def\rvs{{\mathbf{s}}}
\def\rvt{{\mathbf{t}}}
\def\rvu{{\mathbf{u}}}
\def\rvv{{\mathbf{v}}}
\def\rvw{{\mathbf{w}}}
\def\rvx{{\mathbf{x}}}
\def\rvy{{\mathbf{y}}}
\def\rvz{{\mathbf{z}}}

% Elements of random vectors
\def\erva{{\textnormal{a}}}
\def\ervb{{\textnormal{b}}}
\def\ervc{{\textnormal{c}}}
\def\ervd{{\textnormal{d}}}
\def\erve{{\textnormal{e}}}
\def\ervf{{\textnormal{f}}}
\def\ervg{{\textnormal{g}}}
\def\ervh{{\textnormal{h}}}
\def\ervi{{\textnormal{i}}}
\def\ervj{{\textnormal{j}}}
\def\ervk{{\textnormal{k}}}
\def\ervl{{\textnormal{l}}}
\def\ervm{{\textnormal{m}}}
\def\ervn{{\textnormal{n}}}
\def\ervo{{\textnormal{o}}}
\def\ervp{{\textnormal{p}}}
\def\ervq{{\textnormal{q}}}
\def\ervr{{\textnormal{r}}}
\def\ervs{{\textnormal{s}}}
\def\ervt{{\textnormal{t}}}
\def\ervu{{\textnormal{u}}}
\def\ervv{{\textnormal{v}}}
\def\ervw{{\textnormal{w}}}
\def\ervx{{\textnormal{x}}}
\def\ervy{{\textnormal{y}}}
\def\ervz{{\textnormal{z}}}

% Random matrices
\def\rmA{{\mathbf{A}}}
\def\rmB{{\mathbf{B}}}
\def\rmC{{\mathbf{C}}}
\def\rmD{{\mathbf{D}}}
\def\rmE{{\mathbf{E}}}
\def\rmF{{\mathbf{F}}}
\def\rmG{{\mathbf{G}}}
\def\rmH{{\mathbf{H}}}
\def\rmI{{\mathbf{I}}}
\def\rmJ{{\mathbf{J}}}
\def\rmK{{\mathbf{K}}}
\def\rmL{{\mathbf{L}}}
\def\rmM{{\mathbf{M}}}
\def\rmN{{\mathbf{N}}}
\def\rmO{{\mathbf{O}}}
\def\rmP{{\mathbf{P}}}
\def\rmQ{{\mathbf{Q}}}
\def\rmR{{\mathbf{R}}}
\def\rmS{{\mathbf{S}}}
\def\rmT{{\mathbf{T}}}
\def\rmU{{\mathbf{U}}}
\def\rmV{{\mathbf{V}}}
\def\rmW{{\mathbf{W}}}
\def\rmX{{\mathbf{X}}}
\def\rmY{{\mathbf{Y}}}
\def\rmZ{{\mathbf{Z}}}

% Elements of random matrices
\def\ermA{{\textnormal{A}}}
\def\ermB{{\textnormal{B}}}
\def\ermC{{\textnormal{C}}}
\def\ermD{{\textnormal{D}}}
\def\ermE{{\textnormal{E}}}
\def\ermF{{\textnormal{F}}}
\def\ermG{{\textnormal{G}}}
\def\ermH{{\textnormal{H}}}
\def\ermI{{\textnormal{I}}}
\def\ermJ{{\textnormal{J}}}
\def\ermK{{\textnormal{K}}}
\def\ermL{{\textnormal{L}}}
\def\ermM{{\textnormal{M}}}
\def\ermN{{\textnormal{N}}}
\def\ermO{{\textnormal{O}}}
\def\ermP{{\textnormal{P}}}
\def\ermQ{{\textnormal{Q}}}
\def\ermR{{\textnormal{R}}}
\def\ermS{{\textnormal{S}}}
\def\ermT{{\textnormal{T}}}
\def\ermU{{\textnormal{U}}}
\def\ermV{{\textnormal{V}}}
\def\ermW{{\textnormal{W}}}
\def\ermX{{\textnormal{X}}}
\def\ermY{{\textnormal{Y}}}
\def\ermZ{{\textnormal{Z}}}

% Vectors
\def\vzero{{\bm{0}}}
\def\vone{{\bm{1}}}
\def\vmu{{\bm{\mu}}}
\def\vtheta{{\bm{\theta}}}
\def\va{{\bm{a}}}
\def\vb{{\bm{b}}}
\def\vc{{\bm{c}}}
\def\vd{{\bm{d}}}
\def\ve{{\bm{e}}}
\def\vf{{\bm{f}}}
\def\vg{{\bm{g}}}
\def\vh{{\bm{h}}}
\def\vi{{\bm{i}}}
\def\vj{{\bm{j}}}
\def\vk{{\bm{k}}}
\def\vl{{\bm{l}}}
\def\vm{{\bm{m}}}
\def\vn{{\bm{n}}}
\def\vo{{\bm{o}}}
\def\vp{{\bm{p}}}
\def\vq{{\bm{q}}}
\def\vr{{\bm{r}}}
\def\vs{{\bm{s}}}
\def\vt{{\bm{t}}}
\def\vu{{\bm{u}}}
\def\vv{{\bm{v}}}
\def\vw{{\bm{w}}}
\def\vx{{\bm{x}}}
\def\vy{{\bm{y}}}
\def\vz{{\bm{z}}}

% Elements of vectors
\def\evalpha{{\alpha}}
\def\evbeta{{\beta}}
\def\evepsilon{{\epsilon}}
\def\evlambda{{\lambda}}
\def\evomega{{\omega}}
\def\evmu{{\mu}}
\def\evpsi{{\psi}}
\def\evsigma{{\sigma}}
\def\evtheta{{\theta}}
\def\eva{{a}}
\def\evb{{b}}
\def\evc{{c}}
\def\evd{{d}}
\def\eve{{e}}
\def\evf{{f}}
\def\evg{{g}}
\def\evh{{h}}
\def\evi{{i}}
\def\evj{{j}}
\def\evk{{k}}
\def\evl{{l}}
\def\evm{{m}}
\def\evn{{n}}
\def\evo{{o}}
\def\evp{{p}}
\def\evq{{q}}
\def\evr{{r}}
\def\evs{{s}}
\def\evt{{t}}
\def\evu{{u}}
\def\evv{{v}}
\def\evw{{w}}
\def\evx{{x}}
\def\evy{{y}}
\def\evz{{z}}

% Matrix
\def\mA{{\bm{A}}}
\def\mB{{\bm{B}}}
\def\mC{{\bm{C}}}
\def\mD{{\bm{D}}}
\def\mE{{\bm{E}}}
\def\mF{{\bm{F}}}
\def\mG{{\bm{G}}}
\def\mH{{\bm{H}}}
\def\mI{{\bm{I}}}
\def\mJ{{\bm{J}}}
\def\mK{{\bm{K}}}
\def\mL{{\bm{L}}}
\def\mM{{\bm{M}}}
\def\mN{{\bm{N}}}
\def\mO{{\bm{O}}}
\def\mP{{\bm{P}}}
\def\mQ{{\bm{Q}}}
\def\mR{{\bm{R}}}
\def\mS{{\bm{S}}}
\def\mT{{\bm{T}}}
\def\mU{{\bm{U}}}
\def\mV{{\bm{V}}}
\def\mW{{\bm{W}}}
\def\mX{{\bm{X}}}
\def\mY{{\bm{Y}}}
\def\mZ{{\bm{Z}}}
\def\mBeta{{\bm{\beta}}}
\def\mPhi{{\bm{\Phi}}}
\def\mLambda{{\bm{\Lambda}}}
\def\mSigma{{\bm{\Sigma}}}

% Tensor
\DeclareMathAlphabet{\mathsfit}{\encodingdefault}{\sfdefault}{m}{sl}
\SetMathAlphabet{\mathsfit}{bold}{\encodingdefault}{\sfdefault}{bx}{n}
\newcommand{\tens}[1]{\bm{\mathsfit{#1}}}
\def\tA{{\tens{A}}}
\def\tB{{\tens{B}}}
\def\tC{{\tens{C}}}
\def\tD{{\tens{D}}}
\def\tE{{\tens{E}}}
\def\tF{{\tens{F}}}
\def\tG{{\tens{G}}}
\def\tH{{\tens{H}}}
\def\tI{{\tens{I}}}
\def\tJ{{\tens{J}}}
\def\tK{{\tens{K}}}
\def\tL{{\tens{L}}}
\def\tM{{\tens{M}}}
\def\tN{{\tens{N}}}
\def\tO{{\tens{O}}}
\def\tP{{\tens{P}}}
\def\tQ{{\tens{Q}}}
\def\tR{{\tens{R}}}
\def\tS{{\tens{S}}}
\def\tT{{\tens{T}}}
\def\tU{{\tens{U}}}
\def\tV{{\tens{V}}}
\def\tW{{\tens{W}}}
\def\tX{{\tens{X}}}
\def\tY{{\tens{Y}}}
\def\tZ{{\tens{Z}}}


% Graph
\def\gA{{\mathcal{A}}}
\def\gB{{\mathcal{B}}}
\def\gC{{\mathcal{C}}}
\def\gD{{\mathcal{D}}}
\def\gE{{\mathcal{E}}}
\def\gF{{\mathcal{F}}}
\def\gG{{\mathcal{G}}}
\def\gH{{\mathcal{H}}}
\def\gI{{\mathcal{I}}}
\def\gJ{{\mathcal{J}}}
\def\gK{{\mathcal{K}}}
\def\gL{{\mathcal{L}}}
\def\gM{{\mathcal{M}}}
\def\gN{{\mathcal{N}}}
\def\gO{{\mathcal{O}}}
\def\gP{{\mathcal{P}}}
\def\gQ{{\mathcal{Q}}}
\def\gR{{\mathcal{R}}}
\def\gS{{\mathcal{S}}}
\def\gT{{\mathcal{T}}}
\def\gU{{\mathcal{U}}}
\def\gV{{\mathcal{V}}}
\def\gW{{\mathcal{W}}}
\def\gX{{\mathcal{X}}}
\def\gY{{\mathcal{Y}}}
\def\gZ{{\mathcal{Z}}}

% Sets
\def\sA{{\mathbb{A}}}
\def\sB{{\mathbb{B}}}
\def\sC{{\mathbb{C}}}
\def\sD{{\mathbb{D}}}
% Don't use a set called E, because this would be the same as our symbol
% for expectation.
\def\sF{{\mathbb{F}}}
\def\sG{{\mathbb{G}}}
\def\sH{{\mathbb{H}}}
\def\sI{{\mathbb{I}}}
\def\sJ{{\mathbb{J}}}
\def\sK{{\mathbb{K}}}
\def\sL{{\mathbb{L}}}
\def\sM{{\mathbb{M}}}
\def\sN{{\mathbb{N}}}
\def\sO{{\mathbb{O}}}
\def\sP{{\mathbb{P}}}
\def\sQ{{\mathbb{Q}}}
\def\sR{{\mathbb{R}}}
\def\sS{{\mathbb{S}}}
\def\sT{{\mathbb{T}}}
\def\sU{{\mathbb{U}}}
\def\sV{{\mathbb{V}}}
\def\sW{{\mathbb{W}}}
\def\sX{{\mathbb{X}}}
\def\sY{{\mathbb{Y}}}
\def\sZ{{\mathbb{Z}}}

% Entries of a matrix
\def\emLambda{{\Lambda}}
\def\emA{{A}}
\def\emB{{B}}
\def\emC{{C}}
\def\emD{{D}}
\def\emE{{E}}
\def\emF{{F}}
\def\emG{{G}}
\def\emH{{H}}
\def\emI{{I}}
\def\emJ{{J}}
\def\emK{{K}}
\def\emL{{L}}
\def\emM{{M}}
\def\emN{{N}}
\def\emO{{O}}
\def\emP{{P}}
\def\emQ{{Q}}
\def\emR{{R}}
\def\emS{{S}}
\def\emT{{T}}
\def\emU{{U}}
\def\emV{{V}}
\def\emW{{W}}
\def\emX{{X}}
\def\emY{{Y}}
\def\emZ{{Z}}
\def\emSigma{{\Sigma}}

% entries of a tensor
% Same font as tensor, without \bm wrapper
\newcommand{\etens}[1]{\mathsfit{#1}}
\def\etLambda{{\etens{\Lambda}}}
\def\etA{{\etens{A}}}
\def\etB{{\etens{B}}}
\def\etC{{\etens{C}}}
\def\etD{{\etens{D}}}
\def\etE{{\etens{E}}}
\def\etF{{\etens{F}}}
\def\etG{{\etens{G}}}
\def\etH{{\etens{H}}}
\def\etI{{\etens{I}}}
\def\etJ{{\etens{J}}}
\def\etK{{\etens{K}}}
\def\etL{{\etens{L}}}
\def\etM{{\etens{M}}}
\def\etN{{\etens{N}}}
\def\etO{{\etens{O}}}
\def\etP{{\etens{P}}}
\def\etQ{{\etens{Q}}}
\def\etR{{\etens{R}}}
\def\etS{{\etens{S}}}
\def\etT{{\etens{T}}}
\def\etU{{\etens{U}}}
\def\etV{{\etens{V}}}
\def\etW{{\etens{W}}}
\def\etX{{\etens{X}}}
\def\etY{{\etens{Y}}}
\def\etZ{{\etens{Z}}}

% The true underlying data generating distribution
\newcommand{\pdata}{p_{\rm{data}}}
% The empirical distribution defined by the training set
\newcommand{\ptrain}{\hat{p}_{\rm{data}}}
\newcommand{\Ptrain}{\hat{P}_{\rm{data}}}
% The model distribution
\newcommand{\pmodel}{p_{\rm{model}}}
\newcommand{\Pmodel}{P_{\rm{model}}}
\newcommand{\ptildemodel}{\tilde{p}_{\rm{model}}}
% Stochastic autoencoder distributions
\newcommand{\pencode}{p_{\rm{encoder}}}
\newcommand{\pdecode}{p_{\rm{decoder}}}
\newcommand{\precons}{p_{\rm{reconstruct}}}

\newcommand{\laplace}{\mathrm{Laplace}} % Laplace distribution

\newcommand{\E}{\mathbb{E}}
\newcommand{\Ls}{\mathcal{L}}
\newcommand{\R}{\mathbb{R}}
\newcommand{\emp}{\tilde{p}}
\newcommand{\lr}{\alpha}
\newcommand{\reg}{\lambda}
\newcommand{\rect}{\mathrm{rectifier}}
\newcommand{\softmax}{\mathrm{softmax}}
\newcommand{\sigmoid}{\sigma}
\newcommand{\softplus}{\zeta}
\newcommand{\KL}{D_{\mathrm{KL}}}
\newcommand{\Var}{\mathrm{Var}}
\newcommand{\standarderror}{\mathrm{SE}}
\newcommand{\Cov}{\mathrm{Cov}}
% Wolfram Mathworld says $L^2$ is for function spaces and $\ell^2$ is for vectors
% But then they seem to use $L^2$ for vectors throughout the site, and so does
% wikipedia.
\newcommand{\normlzero}{L^0}
\newcommand{\normlone}{L^1}
\newcommand{\normltwo}{L^2}
\newcommand{\normlp}{L^p}
\newcommand{\normmax}{L^\infty}

\newcommand{\parents}{Pa} % See usage in notation.tex. Chosen to match Daphne's book.

\DeclareMathOperator*{\argmax}{arg\,max}
\DeclareMathOperator*{\argmin}{arg\,min}

\DeclareMathOperator{\sign}{sign}
\DeclareMathOperator{\Tr}{Tr}
\let\ab\allowbreak

%\newcommand{\linetofill}{\ \\\hphantom{\hspace{0mm}}\hrulefill\ \\}
\newcommand{\linetofill}{\ \\\hphantom{\hspace{0mm}}\dotfill\ \\}
\newcommand{\vlinetofill}[1]{\\\rule{#1 mm}{0.1mm}\ \\}
\newcommand{\cbox}{
  \setlength{\unitlength}{1pt}
  \begin{picture}(10,10)
    \put(0,0){\line(1,0){8}} \put(0,8){\line(1,0){8}}
    \put(0,0){\line(0,1){8}} \put(8,0){\line(0,1){8}}
  \end{picture}
}
\newcommand{\bigcbox}{
  \setlength{\unitlength}{3pt}
  \begin{picture}(10,10)(1,1)
    \put(0,0){\line(1,0){8}} \put(0,8){\line(1,0){8}}
    \put(0,0){\line(0,1){8}} \put(8,0){\line(0,1){8}}
  \end{picture}
}
% points and boxes on the right
\newcommand{\points}[1]{
  \ \\[-5mm]
  \hphantom{\ }\hfill\textbf{#1 pts \bigcbox \hspace{-6mm}}
  \vspace{5mm}
}
% multiple choice checkboxes
%\newcommand{\boxt}{\hspace*{2em} $[~~]$ \textsf{True}}
%\newcommand{\boxf}{\hspace*{2em} $[~~]$ \textsf{False}}
\newcommand{\boxt}{\hspace*{2em} $\square$ \textsf{True}}
\newcommand{\boxf}{\hspace*{2em} $\square$ \textsf{False}}
\newcommand{\justif}{\textsf{Justification}:  \rule{0ex}{2em}\dotfill\\\rule{0ex}{2em}\dotfill}
\newcommand{\checkboxWithJustification}{\boxt \boxf\\ \justif\\}
\newcommand{\checkbox}{\boxt \boxf}
\usepackage{tikz}
\usepackage{xcolor}
\usepackage{amssymb}
\usepackage{amsmath}
\usepackage{amsthm}
\usepackage{pgfplots}
\pgfmathdeclarefunction{gauss}{2}{%
  \pgfmathparse{1/(#2*sqrt(2*pi))*exp(-((x-#1)^2)/(2*#2^2))}%
}
\renewcommand{\S}{{\cal S}}
\usepackage{pgfplots}
\pgfplotsset{compat=newest}
\pgfplotsset{small, every non boxed x axis/.append style={x axis line style=-},
  every non boxed y axis/.append style={y axis line style=-}}

\newcommand{\timestamp}{\ddmmyyyydate\today \,\,- \currenttime h}
%%% Local Variables:
%%% mode: latex
%%% TeX-master: "exam"
%%% End:

% Numbers
\usepackage[group-separator={,}]{siunitx}

\usepackage{cleveref}
\crefname{section}{\S}{\S\S}
\Crefname{section}{\S}{\S\S}
\crefname{table}{Tab.}{}
\crefname{figure}{Fig.}{}
\crefname{algorithm}{Algorithm}{}
\crefname{equation}{eq.}{}
\crefname{appendix}{App.}{}
\crefname{thm}{Theorem}{}
\crefname{prop}{Proposition}{}
\crefname{cor}{Corollary}{}
\crefname{observation}{Observation}{}
\crefname{assumption}{Assumption}{}
\crefformat{section}{\S#2#1#3}

\usepackage{multirow}
\title{NLP Assignment}
\begin{document}
\setserie{1}


\newcommand{\pair}[2]{{\langle #1 , #2 \rangle}}
\newcommand{\score}[2]{\text{score}_{\theta}(\langle #1, #2 \rangle, \boldsymbol{w})}
\newcommand{\sscore}[1]{\text{score}_{\theta}(#1, \boldsymbol{w})}

\lectureheader{Prof. Ryan Cotterell}
{}
{\Large Natural Language Processing}{Fall 2022}
\begin{center}
	{\Huge Simon Wachter: Assignment 3}\\
	\quad\newline
	siwachte@ethz.ch, 19-920-198\\
	\quad\newline
	\timestamp
\end{center}
\begin{question}\\
	\begin{subquestion}
		We show that under Definition 1, the following holds:
		\begin{align}
			a^* & \overset{!}{=} 1 \oplus a \otimes a^*                        \\
			    & = 1 \oplus a \otimes \bigotimes_{n=0}^{\infty} a^{\otimes n} \\
			    & = 1 \oplus \bigotimes_{n=0}^{\infty} a^{\otimes n+1}         \\
			    & = 1 \oplus \bigotimes_{n=1}^{\infty} a^{\otimes n}           \\
			    & = \bigotimes_{n=0}^{\infty} a^{\otimes n}                    \\
			    & = a^*
		\end{align}
	\end{subquestion}
	\begin{subquestion}
		First we show that $\log a \oplus a = \log(2) + a$:
		\begin{align}
			a \oplus \log a & = \log(e^{a} + e^{a}) \\
			                & = \log(2 e^{a})       \\
			                & = \log(2) + a
		\end{align}
		Then we calculate the Kleene start:
		\begin{align}
			\bigoplus_{\log \; n=0}^{\infty} a^{\oplus n} & = a^{\oplus 0} \oplus_{\log} \left( \bigoplus_{\log \; n=1}^{\infty} a^{\oplus n} \right)   \\
			                                              & = 0 \oplus_{\log} \left( \bigoplus_{\log \; n=1}^{\infty} a^{\oplus n} \right)              \\
			                                              & = 0 \oplus_{\log} a \oplus_{\log} 2a \oplus_{\log} 3a \oplus_{\log}\cdots                   \\
			                                              & = \log\left( e^{0} + e^{a} \right) \oplus_{\log} 2a \oplus_{\log} 3a \oplus_{\log}\cdots    \\
			                                              & = \log \left( e^{\log(e^{0} + e^{a})} + e^{2a} \right) \oplus_{\log} 3a \oplus_{\log}\cdots \\
			                                              & = \log \left( e^{0} + e^{a} + e^{2a} \right) \oplus_{\log} 3a \oplus_{\log}\cdots           \\
			                                              & = \log \left( \sum_{n=0}^{\infty} e^{a^{\oplus n}} \right)
		\end{align}
		We have two cases here, either $a > 0$ or $a \leq 0$. In the first case, the sum diverges and we get $\infty$. In the second case, the sum converges:
		\begin{align}
			\sum_{n=0}^{\infty} e^{a^{\oplus n}} & = \frac{1}{1 - e^{a}} &  & \text{limit geometric series}
		\end{align}
		Therefore, we get:
		\begin{align}
			\log \left( \sum_{n=0}^{\infty} e^{a^{\oplus n}} \right) & = \log \left( \frac{1}{1 - e^{a}}\right) \\
			                                                         & = \log(1) - \log(1 - e^{a})              \\
			                                                         & = \log(1 - e^{a})
		\end{align}
	\end{subquestion}
	\begin{subquestion}
		First we derive a closed form solution for $(x, y)^{\oplus n}$:
		\begin{align}
			\langle x, y \rangle^{\oplus 1} & = \langle x, y \rangle                                              \\
			\langle x, y \rangle^{\oplus 2} & = \langle x^2, 2xy \rangle                                          \\
			\langle x, y \rangle^{\oplus 3} & = \langle x^3, 3x^2y \rangle                                        \\
			\langle x, y \rangle^{\oplus 4} & = \langle x^4, 4x^3y \rangle                                        \\
			\langle x, y \rangle^{\oplus n} & = \langle x^n, nx^{n-1}y \rangle \label{eq:power_oplus_expectation}
		\end{align}
		Then we derive a closed form for $a^*$:
		\begin{align}
			\bigoplus_{n=0}^{\infty} \langle x, y \rangle^{\oplus n} & = \bigoplus_{n=0}^{\infty} \langle x^n, nx^{n-1}y \rangle                             & \text{\cref*{eq:power_oplus_expectation}} \\
			                                                         & = \left\langle \sum_{n=0}^{\infty} x^{n}, \sum_{n=0}^{\infty} nx^{n-1} y\right\rangle
		\end{align}
		Both parts only converge for $|x| < 1$. The left parts is a geometric series and has limit $\frac{1}{1-x}$. And the right parts, which is a power series:
		\begin{align}
			\sum_{n=0}^{\infty} nx^{n-1} y & = y \sum_{n=0}^{\infty} nx^{n-1}                                                  \\
			                               & = y \left( 0 + \sum_{n=1}^{\infty} nx^{n-1}\right)                                \\
			                               & = y \sum_{n=0}^{\infty} (n+1) x^{n}                                               \\
			                               & = y \frac{1}{(x -1)^2}                             &  & \text{limit power series} \\
			                               & = \frac{y}{(x-1)^2}
		\end{align}
		Hence our closed form with inserted limits is given by:
		\begin{align}
			\left\langle \sum_{n=0}^{\infty} x^{n}, \sum_{n=0}^{\infty} nx^{n-1} y\right\rangle & = \left\langle \frac{1}{1-x}, \frac{y}{(x-1)^2}\right\rangle &  & |x| \leq 1
		\end{align}
		If $|x| > 1$ both parts diverge.
	\end{subquestion}
	\begin{subquestion}
		\begin{align}
			\mathcal{W}_{\text{lang}} & =\left\langle 2^{\Sigma^*}, \bigcup, \otimes, \{\}, \{\epsilon\} \right\rangle
		\end{align}
		We first show that $\mathcal{W}_{\text{lang}}$ is a semiring:
		\begin{itemize}
			\item  $(2^{\Sigma^*}, \oplus, \mathbf{0})$ must be a commutative monoid with identity element $\mathbf{0}$:
			      \begin{align}
				      \left(x \oplus y\right) \oplus z & = (x \cup y) \cup z                \\
				                                       & = \{x, y\} \cup z                  \\
				                                       & = \{x, y, z\}                      \\
				                                       & = x \oplus \{y, z\}                \\
				                                       & = x \oplus \left(y \oplus z\right)
			      \end{align}
			      \begin{align}
				      \mathbf{0} \oplus x & = \{\} \oplus x \\
				                          & = \{\} \cup x   \\
				                          & = \oplus
			      \end{align}
			      \begin{align}
				      x \oplus y & = x \cup y \\
				                 & = \{x, y\} \\
				                 & = y \cup x \\
				                 & = y + x
			      \end{align}
			\item $(2^{\Sigma^*}, \otimes, \mathbf{1})$ must be a monoid with identity element $\mathbf{1}$:
			      \begin{align}
				      \left( x \otimes y \right) \otimes z & = xy \otimes z                       \\
				                                           & = xyz                                \\
				                                           & = x \otimes yz                       \\
				                                           & = x \otimes \left(y \otimes z\right)
			      \end{align}
			      \begin{align}
				      \mathbf{1} \otimes x & = \{\epsilon\} \otimes x \\
				                           & = x                      \\
				                           & = x \otimes \{\epsilon\} \\
				                           & = x \otimes \mathbf{1}
			      \end{align}
			\item Multiplication left and right distributes over addition:
			      \begin{align}
				      x \otimes \left( y \oplus z\right) & = x \otimes \{y, z\}                                          \\
				                                         & = \{xy, xz\}                                                  \\
				                                         & = \{xy\} \cup \{xz\}                                          \\
				                                         & = \left( x \otimes y \right) \oplus \left( x \otimes z\right)
			      \end{align}
			      \begin{align}
				      \left(x \oplus y\right) \otimes z & = \{x, y\} \otimes z                                         \\
				                                        & = \{xz, yz\}                                                 \\
				                                        & = \{xz\} \cup \{yz\}                                         \\
				                                        & = \left( x \otimes z\right) \oplus \left( y \otimes z\right)
			      \end{align}
			\item Multiplication by $\mathbf{0}$ annihilates $\R \times \R$:
			      \begin{align}
				      \mathbf{0} \otimes x & = \{a \circ b \,|\, a \in A, b \in \{\}\}                                                             \\
				                           & = \{\}                                    &  & \text{by definition of $\circ$, because no $b$ exists} \\
				                           & = x \otimes \mathbf{0}                                                                                \\
			      \end{align}
		\end{itemize}
		The Kleene start for $\mathcal{W}_{\text{lang}}$ given by:
		\begin{align}
			A^{\otimes n} & = \{ \bigotimes_{i=0}^{n} a_i \;|\; a_i \in A\}                                     \\
			A^*           & = \bigoplus_{n=0}^{\infty} A^{\otimes n}                                            \\
			              & = \bigoplus_{n=0}^{\infty} \left\{ \bigotimes_{i=0}^{n} a_i \;|\; a_i \in A\right\} \\
			              & = \left\{ \bigotimes_{i=0}^{n} a_i \;|\; a_i \in A, n \in \mathbb{Z} \right\}
		\end{align}
	\end{subquestion}
\end{question}
\begin{question}\\
	\begin{subquestion}
		Tropical semiring is $0$-closed:
		\begin{align}
			a \oplus \boldsymbol{0} & = \min(a, \boldsymbol{0})                                                 \\
			                        & = \min(a, 0)                                                              \\
			                        & = 0                       &  & \text{because $a \in \mathbb{R}_{\geq 0}$}
		\end{align}
		Arctic semiring is $0$-closed:
		\begin{align}
			a \oplus \boldsymbol{0} & = \max(a, \boldsymbol{0})                                                 \\
			                        & = \max(a, 0)                                                              \\
			                        & = 0                       &  & \text{because $a \in \mathbb{R}_{\leq 0}$}
		\end{align}
	\end{subquestion}
	\begin{subquestion}
		Proof by induction:\\
		Base case $i = 1$:
		\begin{align}
			M^1    & = M                             \\
			M_{ij} & = w_{ij} & \text{by def of $M$}
		\end{align}
		$w_{ij}$ is exactly the semiring-sum over all paths from $i$ to $j$ of length $1$. This holds because there is only one path of length $1$ from $i$ to $j$.\\
		Induction hypothesis:
		$M^{i}_{ij}$ is the semiring-sum over all paths from $i$ to $j$ of length $i$.\\
		Induction step $i \rightarrow i + 1$:\\
		\begin{align}
			M^{i + 1}    & = M^i \otimes M                                                                             \\
			M^{i+1}_{kj} & = \sum_{l=0}^{n} M^{i}_{kl} \otimes M_{lj} & \text{def. matrix mult.} \label{eq:mip1_entry}
		\end{align}
		In \cref*{eq:mip1_entry} we sum over the product of all possible paths of length $i$ from $k$ to another node $l$ and all possible paths of length $1$ from nodes $l$ to $j$. This sum is exactly the semiring-sum over all possible paths of length $i + 1$ from $k$ to $j$.\\
	\end{subquestion}
	\begin{subquestion}If we have a graph $G$ with $N$ vertices, then a path with length $l \geq N$ must visit at least one vertex $v$ twice. From this follows, that we can just remove this cycle at vertex $v$ and reduce the path to a path of length at most $N - 1$. If there are multiple cycles, we can remove all of them until we arrive at a path of length at most $N - 1$. Let us now assume that the longest path in $G$ has length $l \geq N-1$. The weight of this path is the following:
		\begin{align}
			v_{0} \overset{w_0}{\rightarrow} v_1 \rightarrow \dots \rightarrow v_{k} \rightarrow \dots \rightarrow v_{k} \rightarrow \dots \overset{v_{l-2}}{\rightarrow} v_{l-1} \overset{w_{l-1}}{\rightarrow} v_l \\
			w_0 \otimes \dots \otimes w_{k-1} \otimes w_{k} \otimes \dots \otimes w'_{k} \otimes w_{k+1} \otimes \dots \otimes w_{l-1}
		\end{align}
		Notice the cycle in the middle, which we know exists given our reasoning before. We can remove this cycle and arrive at the following path with new weight:
		\begin{align}
			v_{0} \overset{w_0}{\rightarrow} v_1 \rightarrow \dots \rightarrow v_{k} \rightarrow \dots \overset{v_{l-2}}{\rightarrow} v_{l-1} \overset{w_{l-1}}{\rightarrow} v_l \\
			w_0 \otimes \dots \otimes w_{k-1} \otimes w_{k+1} \otimes \dots \otimes w_{l-1}
		\end{align}
		We first define weights for subpaths:
		\begin{align}
			s_0 & = w_0 \otimes \dots \otimes w_{k-1}     &  & \text{path to $k$}            \\
			s_1 & = w_{k} \otimes \dots \otimes w'_{k}    &  & \text{cycle at $k$-th vertex} \\
			s_2 & = w_{k+1} \otimes \dots \otimes w_{l-1} &  & \text{path from $k$ to $l$}   \\
		\end{align}
		We now rewright out path weights and add them over our semiring:
		\begin{align}
			\left( s_0 \otimes s_1 \otimes s_2 \right) \oplus \left( s_0 \otimes s_2 \right)                       \\
			\left( s_0 \right) \otimes \left( \left( s_1 \otimes s_2 \right) \oplus s_2 \right)                    \\
			\left( s_0 \right) \otimes \left( s_2 \otimes \left( s_1 \oplus 1 \right) \right)                      \\
			\left( s_0 \right) \otimes \left( s_2 \otimes 1 \right) &  & \text{def. $0$-closed}                    \\
			s_0 \otimes s_2                                         &  & \text{def. $\mathbf{1}$} \label{eq:end2c}
		\end{align}
		We see that \cref{eq:end2c} is exactly the weight of the path without the cycle. Since this holds for every path of length $l \geq N$ we can conclude that the longest path in $G$ has length at most $N - 1$.\\
	\end{subquestion}
	\begin{subquestion}
		Per definition of the Kleene start, we have:
		\begin{align}
			M^* & = \bigoplus_{i=0}^{\infty} M^{\otimes i}
		\end{align}
		In b) we have shown that $M^{\otimes i}$ is the semiring-sum over all paths of length $i$ and in c) we have shown that the shortest path depends only on paths of length $l \leq N-1$. We also showed that under the $\bigoplus$ operation, only paths of length $l \leq N-1$ are considered. Therefore, we know that $M^*$ depends only on:
		\begin{align}
			M^{*} & = \bigoplus_{i=0}^{N-1} M^{\otimes i}
		\end{align}
	\end{subquestion}
	\begin{subquestion}
		We define a simple algorithm:\\
		\begin{algorithm}[H]
			\label{M_star}
			\caption{Matrix multiplication for Kleene star}
			$M$\;
			$M' \gets M$\;
			$M^* \gets \mathbf{0}$\;
			\For{$i = 0$ to $N-1$}{
			\For{$j = 0$ to len($M$)}{
			\For{$k = 0$ to len($M$)}{
			\For{$l = 0$ to len($M$)}{
			$M'_{j,k} \gets M'_{j,k} \oplus \left( M_{j,l} \otimes M_{l,k}\right)$
			}
			$M^*_{j,k} \gets M^*_{j,k} \oplus M'_{j,k}$
			}
			}
			$M \gets M'$\;
			}
		\end{algorithm}
		The inner for loops calculate the matrix multiplication, $M^{\otimes n}$, and the outer loop iterates $N-1$ times to calculate the Kleene star. Since each loop has $\mathcal{O}(N)$ iterations, the algorithm has a runtime of $\mathcal{O}(N^4)$.\\
	\end{subquestion}
	\begin{subquestion}
		\begin{align}
			a \oplus a & = a \otimes \left( \mathbf{1} \oplus \mathbf{1} \right)                               \\
			           & = a \otimes \mathbf{1}                                  &  & \text{def. 0-closed}     \\
			           & = a                                                     &  & \text{def. $\mathbf{1}$}
		\end{align}
	\end{subquestion}
	\begin{subquestion}
		We show given equality with an induction proof:
		Base case $n = 0$:
		\begin{align}
			\bigoplus_{n=0}^{0} M^n & = M^0                         \\
			                        & = \left( I \oplus M \right)^0
		\end{align}
		Where we assumed that:
		\begin{align}
			I \oplus M = M
		\end{align}
		Our induction hypothesis is:
		\begin{align}
			\bigoplus_{n=0}^{n} M^n = \left( I \oplus M \right)^{n}
		\end{align}
		Now for the inductive step we have $n \rightarrow n+1$:
		\begin{align}
			\bigoplus_{i=0}^{n+1} M^i & = \bigoplus_{i=0}^{n} M^i \oplus M^{n+1}                                                                         \\
			                          & = M^0 \oplus \bigoplus_{i=0}^{n} M^i \oplus M^{n+1}                                                              \\
			                          & = M^0 \oplus\bigoplus_{i=0}^{n} \left(  M^i \oplus M^i \right) \oplus M^{n+1}    &  & \text{(def. Idempotent)}   \\
			                          & = \bigoplus_{i=0}^{n} M^i \oplus \bigoplus_{i=1}^{n+1} M^i                                                       \\
			                          & = \bigoplus_{i=0}^{n} M^{i} \otimes \left( \mathbf{I} \oplus M\right)            &  & \text{(def. distributive)} \\
			                          & = \left( \mathbf{I} \oplus M \right)^n \otimes \left( \mathbf{I} \oplus M\right) &  & \text{(def. I.H.)}         \\
			                          & = \left( \mathbf{I} \oplus M \right)^{n+1}
		\end{align}
	\end{subquestion}
	\begin{subquestion}
		With the log factor, we immediately think about binary representation. We use the product of power rule to rewrite our left side of the equation:
		\begin{align}
			\bigotimes_{k=0}^{\left\lfloor \log_2 n \right\rfloor} M^{\alpha_k 2^k} & = M^{\sum_{k=0}^{\left\lfloor \log_2 n \right\rfloor} \alpha_k 2^k}
		\end{align}
		We now analyze the exponent of $M$ more closely. If we choose $\alpha_k$ to represent the k-th bit in the binary representation of $n$, we can see that $\sum_{k=0}^{\left\lfloor \log_2 n \right\rfloor}\alpha_k 2^k = n$. Therefore, we can rewrite the equation as:
		\begin{align}
			\bigotimes_{k=0}^{\left\lfloor \log_2 n \right\rfloor} M^{\alpha_k 2^k} & = M^{\sum_{k=0}^{\left\lfloor \log_2 n \right\rfloor} \alpha_k 2^k} \\
			                                                                        & = M^n
		\end{align}
		With this insight, we can rewrite the algorithm to calculate $M^*$:\\
		\begin{algorithm}[H]
			\label{M_star}
			\caption{Matrix multiplication for Kleene star}
			$M$\;
			$M' \gets M$\;
			$M^* \gets \mathbf{0}$\;
			\For{$i = 0$ to $\log_2(N-1)$}{
			\If{$a_i == 0$}{
			\For{$j = 0$ to len($M$)}{
			\For{$k = 0$ to len($M$)}{
			\For{$l = 0$ to len($M$)}{
			$M'_{j,k} \gets M'_{j,k} \oplus \left( M_{j,l} \otimes M_{l,k}\right)$
			}
			$M^*_{j,k} \gets M^*_{j,k} \oplus M'_{j,k}$
			}
			}
			$M \gets M'$\;
			}
			}
		\end{algorithm}
		As we have just remove some iterations from the outer most loop, our runtime changes to $\mathcal{O}(n^3 \log n)$
	\end{subquestion}
	\begin{subquestion}
		We derive the SVD of $A$:
		\begin{align}
			A       & = U \Sigma V^T                      \\
			||A||_2 & = ||U \Sigma V^T||_2 = ||\Sigma||_2
		\end{align}
		Further we can rewrite the operator norm by w.l.o.g choosing $x$ such that $||x||_2 = 1$:
		\begin{align}
			||A||_2 & = \sup_{x \neq 0} \frac{||Ax||_2}{||x||_2} \\
			        & = \sup_{x \neq 0, ||x||_2 = 1} ||Ax||_2    \\
		\end{align}
		Combining these two insight we get:
		\begin{align}
			||A||_2 & = \sup_{x \neq 0, ||x||_2 = 1} ||\Sigma x||_2                               \\
			        & =\sigma_{\text{max}}(A)                       &  & \text{(min-max theorem)}
		\end{align}
	\end{subquestion}
	\begin{subquestion}
		\begin{align}
			||A^* - \sum_{n=0}^K A^n ||_2 & = || \sum_{n=0}^{\infty} A^n - \sum_{n=0}^{K} A^n ||_2                                                                                            \\
			                              & = || \sum_{n=K+1}^{\infty} A^n ||_2                                                                                                               \\
			                              & = \sigma_{\text{max}}\left(\sum_{n=K+1}^{\infty} A^n \right)      &  & \text{(ex. i))}                                                            \\
			                              & \leq \sum_{n=K+1}^{\infty} \sigma_{\text{max}} \left( A^n \right) &  & \text{(singular value inequalities)}                                       \\
			                              & \leq \sum_{n=K+1}^{\infty} \sigma_{\text{max}} \left( A \right)^n &  & \text{(singular value inequalities)}                                       \\
			                              & = \frac{\sigma_{\text{max}}(A)^{K+1}}{1 - \sigma_{\text{max}}(A)} &  & \text{(geom. series if $\sigma_{\text{max}}(A) < 1$))} \label{eq:closed_j} \\
		\end{align}
		\cref*{eq:closed_j} shows the closed form solution if $\sigma_{\text{max}}(A) < 1$. If $\sigma_{\text{max}}(A) \geq 1$, the closed form solution is not defined as the geometric series diverges.\\
		The condition on $\sigma_{\text{max}}(A)$ for the truncation error to converge to $A^*$ therefore is $\sigma_{\text{max}}(A) < 1$:
		\begin{align}
			\lim_{K \to \infty} \frac{\sigma_{\text{max}}(A)^{K+1}}{1 - \sigma_{\text{max}}(A)} & = 0
		\end{align}
	\end{subquestion}
	\begin{subquestion}
		If $\sigma_{\text{max}}(A) \geq 1$ the truncation error is unbounded and the truncation is not a good approximation to asteration.\\
		In the case where $\sigma_{\text{max}}(A) < 1$ the truncation error in $\mathcal{O}$ is given by:
		\begin{align}
			\mathcal{O}\left( \frac{\sigma_{\text{max}}(A)^{K+1}}{1 - \sigma_{\text{max}}(A)} \right) & = \mathcal{O} \left( \sigma_{\text{max}}(A)^K \right)
		\end{align}
		Hence the error decays exponentially, which is then generally deemed to be acceptable for an error and so a truncation is a good approximation to asteration.
	\end{subquestion}
\end{question}
\end{document}

%%% Local Variables:
%%% mode: latex
%%% TeX-master: t
%%% End: