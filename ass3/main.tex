\documentclass[a4paper,12pt]{ETHexercise}
\usepackage{bbm}

\input{preamble}

\usepackage{multirow}
\title{NLP Assignment}
\begin{document}
\setserie{1}


\newcommand{\pair}[2]{{\langle #1 , #2 \rangle}}
\newcommand{\score}[2]{\text{score}_{\theta}(\langle #1, #2 \rangle, \boldsymbol{w})}
\newcommand{\sscore}[1]{\text{score}_{\theta}(#1, \boldsymbol{w})}

\lectureheader{Prof. Ryan Cotterell}
{}
{\Large Natural Language Processing}{Fall 2022}
\begin{center}
	{\Huge Simon Wachter: Assignment 3}\\
	\quad\newline
	siwachte@ethz.ch, 19-920-198\\
	\quad\newline
	\timestamp
\end{center}
\begin{question}\\
	\begin{subquestion}
		We show that under Definition 1, the following holds:
		\begin{align}
			a^* & \overset{!}{=} 1 \oplus a \otimes a^*                        \\
			    & = 1 \oplus a \otimes \bigotimes_{n=0}^{\infty} a^{\otimes n} \\
			    & = 1 \oplus \bigotimes_{n=0}^{\infty} a^{\otimes n+1}         \\
			    & = 1 \oplus \bigotimes_{n=1}^{\infty} a^{\otimes n}           \\
			    & = \bigotimes_{n=0}^{\infty} a^{\otimes n}                    \\
			    & = a^*
		\end{align}
	\end{subquestion}
	\begin{subquestion}
		\begin{align}
			\mathbb{R}^*
		\end{align}
	\end{subquestion}
	\begin{subquestion}
		\begin{align}
			\left(\mathbb{R} \times \mathbb{R} \right)^*
		\end{align}
	\end{subquestion}
	\begin{subquestion}
		\begin{align}
			\mathcal{W}_{\text{lang}} & =\left\langle 2^{\Sigma^*}, \bigcup, \otimes, \{\}, \{\epsilon\} \right\rangle
		\end{align}
		\begin{itemize}
			\item  $(2^{\Sigma^*}, \oplus, \mathbf{0})$ must be a commutative monoid with identity element $\mathbf{0}$:
			      \begin{align}
				      \left(x \oplus y\right) \oplus z & = (x \cup y) \cup z                \\
				                                       & = \{x, y\} \cup z                  \\
				                                       & = \{x, y, z\}                      \\
				                                       & = x \oplus \{y, z\}                \\
				                                       & = x \oplus \left(y \oplus z\right)
			      \end{align}
			      \begin{align}
				      \mathbf{0} \oplus x & = \{\} \oplus x \\
				                          & = \{\} \cup x   \\
				                          & = \oplus
			      \end{align}
			      \begin{align}
				      x \oplus y & = x \cup y \\
				                 & = \{x, y\} \\
				                 & = y \cup x \\
				                 & = y + x
			      \end{align}
			\item $(2^{\Sigma^*}, \otimes, \mathbf{1})$ must be a monoid with identity element $\mathbf{1}$:
			      \begin{align}
				      \left( x \otimes y \right) \otimes z & = xy \otimes z                       \\
				                                           & = xyz                                \\
				                                           & = x \otimes yz                       \\
				                                           & = x \otimes \left(y \otimes z\right)
			      \end{align}
			      \begin{align}
				      \mathbf{1} \otimes x & = \{\epsilon\} \otimes x \\
				                           & = x                      \\
				                           & = x \otimes \{\epsilon\} \\
				                           & = x \otimes \mathbf{1}
			      \end{align}
			\item Multiplication left and right distributes over addition:
			      \begin{align}
				      x \otimes \left( y \oplus z\right) & = x \otimes \{y, z\}                                          \\
				                                         & = \{xy, xz\}                                                  \\
				                                         & = \{xy\} \cup \{xz\}                                          \\
				                                         & = \left( x \otimes y \right) \oplus \left( x \otimes z\right)
			      \end{align}
			      \begin{align}
				      \left(x \oplus y\right) \otimes z & = \{x, y\} \otimes z                                         \\
				                                        & = \{xz, yz\}                                                 \\
				                                        & = \{xz\} \cup \{yz\}                                         \\
				                                        & = \left( x \otimes z\right) \oplus \left( y \otimes z\right)
			      \end{align}
			\item Multiplication by $\mathbf{0}$ annihilates $\R \times \R$:
			      \begin{align}
				      \mathbf{0} \otimes x & = \{a \circ b \,|\, a \in A, b \in \{\}\}                                                             \\
				                           & = \{\}                                    &  & \text{by definition of $\circ$, because no $b$ exists} \\
				                           & = x \otimes \mathbf{0}                                                                                \\
			      \end{align}
		\end{itemize}
	\end{subquestion}
\end{question}
\begin{question}\\
	\begin{subquestion}
		Tropical semiring is $0$-closed:
		\begin{align}
			a + \mathcal{1} & = a + \infty \\
			                & = \infty
		\end{align}
		Arctic semiring is $0$-closed:
		\begin{align}
			a + \mathcal{1} & = a -\infty \\
			                & = -\infty
		\end{align}
	\end{subquestion}
	\begin{subquestion}
		Proof by induction:\\
		Base case $i = 1$:
		\begin{align}
			M^1    & = M                             \\
			M_{ij} & = w_{ij} & \text{by def of $M$}
		\end{align}
		$w_{ij}$ is exactly the semiring-sum over all paths from $i$ to $j$ of length $1$. This holds because there is only one path of length $1$ from $i$ to $j$.\\
		Induction hypothesis:
		$M^{i}_{ij}$ is the semiring-sum over all paths from $i$ to $j$ of length $i$.\\
		Induction step $i \rightarrow i + 1$:\\
		\begin{align}
			M^{i + 1}    & = M^i \otimes M                                                                            \\
			M^{i+1}_{kj} & = \sum_{l=0}^{n} M^{i}_{kl} \otimes M_{lj} & \text{def matrix mult.} \label{eq:mip1_entry} \\
		\end{align}
		In \cref*{eq:mip1_entry} we sum over the product of all possible paths of length $i$ from $k$ to another node $l$ and all possible paths of length $1$ from nodes $l$ to $j$. This sum is exactly the semiring-sum over all possible paths of length $i + 1$ from $k$ to $j$.\\
	\end{subquestion}
\end{question}
\end{document}

%%% Local Variables:
%%% mode: latex
%%% TeX-master: t
%%% End: