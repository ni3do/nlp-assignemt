\documentclass[a4paper,12pt]{ETHexercise}
\usepackage{bbm}

\input{preamble}

\usepackage{multirow}
\title{NLP Assignment}
\begin{document}
\setserie{1}


\newcommand{\pair}[2]{{\langle #1 , #2 \rangle}}
\newcommand{\score}[2]{\text{score}_{\theta}(\langle #1, #2 \rangle, \boldsymbol{w})}
\newcommand{\sscore}[1]{\text{score}_{\theta}(#1, \boldsymbol{w})}

\lectureheader{Prof. Ryan Cotterell}
{}
{\Large Natural Language Processing}{Fall 2022}
\begin{center}
	{\Huge Simon Wachter: Assignment 3}\\
	\quad\newline
	siwachte@ethz.ch, 19-920-198\\
	\quad\newline
	\timestamp
\end{center}
\begin{question}\\
	\begin{subquestion}
		We show that under Definition 1, the following holds:
		\begin{align}
			a^* & \overset{!}{=} 1 \oplus a \otimes a^*                        \\
			    & = 1 \oplus a \otimes \bigotimes_{n=0}^{\infty} a^{\otimes n} \\
			    & = 1 \oplus \bigotimes_{n=0}^{\infty} a^{\otimes n+1}         \\
			    & = 1 \oplus \bigotimes_{n=1}^{\infty} a^{\otimes n}           \\
			    & = \bigotimes_{n=0}^{\infty} a^{\otimes n}                    \\
			    & = a^*
		\end{align}
	\end{subquestion}
	\begin{subquestion}
		First we show that $\log a \oplus a = \log(2) + a$:
		\begin{align}
			a \oplus \log a & = \log(e^{a} + e^{a}) \\
			                & = \log(2 e^{a})       \\
			                & = \log(2) + a
		\end{align}
		Then we calculate the Kleene start:
		\begin{align}
			\bigoplus_{\log \; n=0}^{\infty} a^{\oplus n} & = a^{\oplus 0} \oplus_{\log} \left( \bigoplus_{\log \; n=1}^{\infty} a^{\oplus n} \right)   \\
			                                              & = 0 \oplus_{\log} \left( \bigoplus_{\log \; n=1}^{\infty} a^{\oplus n} \right)              \\
			                                              & = 0 \oplus_{\log} a \oplus_{\log} 2a \oplus_{\log} 3a \oplus_{\log}\cdots                   \\
			                                              & = \log\left( e^{0} + e^{a} \right) \oplus_{\log} 2a \oplus_{\log} 3a \oplus_{\log}\cdots    \\
			                                              & = \log \left( e^{\log(e^{0} + e^{a})} + e^{2a} \right) \oplus_{\log} 3a \oplus_{\log}\cdots \\
			                                              & = \log \left( e^{0} + e^{a} + e^{2a} \right) \oplus_{\log} 3a \oplus_{\log}\cdots           \\
			                                              & = \log \left( \sum_{n=0}^{\infty} e^{a^{\oplus n}} \right)
		\end{align}
		We have two cases here, either $a > 0$ or $a \leq 0$. In the first case, the sum diverges and we get $\infty$. In the second case, the sum converges:
		\begin{align}
			\sum_{n=0}^{\infty} e^{a^{\oplus n}} & = \frac{1}{1 - e^{a}} &  & \text{limit geometric series}
		\end{align}
		Therefore, we get:
		\begin{align}
			\log \left( \sum_{n=0}^{\infty} e^{a^{\oplus n}} \right) & = \log \left( \frac{1}{1 - e^{a}}\right) \\
			                                                         & = \log(1) - \log(1 - e^{a})              \\
			                                                         & = \log(1 - e^{a})
		\end{align}
	\end{subquestion}
	\begin{subquestion}
		First we derive a closed form solution for $(x, y)^{\oplus n}$:
		\begin{align}
			\langle x, y \rangle^{\oplus 1} & = \langle x, y \rangle                                              \\
			\langle x, y \rangle^{\oplus 2} & = \langle x^2, 2xy \rangle                                          \\
			\langle x, y \rangle^{\oplus 3} & = \langle x^3, 3x^2y \rangle                                        \\
			\langle x, y \rangle^{\oplus 4} & = \langle x^4, 4x^3y \rangle                                        \\
			\langle x, y \rangle^{\oplus n} & = \langle x^n, nx^{n-1}y \rangle \label{eq:power_oplus_expectation}
		\end{align}
		Then we derive a closed form for $a^*$:
		\begin{align}
			\bigoplus_{n=0}^{\infty} \langle x, y \rangle^{\oplus n} & = \bigoplus_{n=0}^{\infty} \langle x^n, nx^{n-1}y \rangle                             & \text{\cref*{eq:power_oplus_expectation}} \\
			                                                         & = \left\langle \sum_{n=0}^{\infty} x^{n}, \sum_{n=0}^{\infty} nx^{n-1} y\right\rangle
		\end{align}
		Both parts only converge for $|x| < 1$. The left parts is a geometric series and has limit $\frac{1}{1-x}$. And the right parts, which is a power series:
		\begin{align}
			\sum_{n=0}^{\infty} nx^{n-1} y & = y \sum_{n=0}^{\infty} nx^{n-1}                                                  \\
			                               & = y \left( 0 + \sum_{n=1}^{\infty} nx^{n-1}\right)                                \\
			                               & = y \sum_{n=0}^{\infty} (n+1) x^{n}                                               \\
			                               & = y \frac{1}{(x -1)^2}                             &  & \text{limit power series} \\
			                               & = \frac{y}{(x-1)^2}
		\end{align}
		Hence our closed form with inserted limits is given by:
		\begin{align}
			\left\langle \sum_{n=0}^{\infty} x^{n}, \sum_{n=0}^{\infty} nx^{n-1} y\right\rangle & = \left\langle \frac{1}{1-x}, \frac{y}{(x-1)^2}\right\rangle &  & |x| \leq 1
		\end{align}
		If $|x| > 1$ both parts diverge.
	\end{subquestion}
	\begin{subquestion}
		\begin{align}
			\mathcal{W}_{\text{lang}} & =\left\langle 2^{\Sigma^*}, \bigcup, \otimes, \{\}, \{\epsilon\} \right\rangle
		\end{align}
		We first show that $\mathcal{W}_{\text{lang}}$ is a semiring:
		\begin{itemize}
			\item  $(2^{\Sigma^*}, \oplus, \mathbf{0})$ must be a commutative monoid with identity element $\mathbf{0}$:
			      \begin{align}
				      \left(x \oplus y\right) \oplus z & = (x \cup y) \cup z                \\
				                                       & = \{x, y\} \cup z                  \\
				                                       & = \{x, y, z\}                      \\
				                                       & = x \oplus \{y, z\}                \\
				                                       & = x \oplus \left(y \oplus z\right)
			      \end{align}
			      \begin{align}
				      \mathbf{0} \oplus x & = \{\} \oplus x \\
				                          & = \{\} \cup x   \\
				                          & = \oplus
			      \end{align}
			      \begin{align}
				      x \oplus y & = x \cup y \\
				                 & = \{x, y\} \\
				                 & = y \cup x \\
				                 & = y + x
			      \end{align}
			\item $(2^{\Sigma^*}, \otimes, \mathbf{1})$ must be a monoid with identity element $\mathbf{1}$:
			      \begin{align}
				      \left( x \otimes y \right) \otimes z & = xy \otimes z                       \\
				                                           & = xyz                                \\
				                                           & = x \otimes yz                       \\
				                                           & = x \otimes \left(y \otimes z\right)
			      \end{align}
			      \begin{align}
				      \mathbf{1} \otimes x & = \{\epsilon\} \otimes x \\
				                           & = x                      \\
				                           & = x \otimes \{\epsilon\} \\
				                           & = x \otimes \mathbf{1}
			      \end{align}
			\item Multiplication left and right distributes over addition:
			      \begin{align}
				      x \otimes \left( y \oplus z\right) & = x \otimes \{y, z\}                                          \\
				                                         & = \{xy, xz\}                                                  \\
				                                         & = \{xy\} \cup \{xz\}                                          \\
				                                         & = \left( x \otimes y \right) \oplus \left( x \otimes z\right)
			      \end{align}
			      \begin{align}
				      \left(x \oplus y\right) \otimes z & = \{x, y\} \otimes z                                         \\
				                                        & = \{xz, yz\}                                                 \\
				                                        & = \{xz\} \cup \{yz\}                                         \\
				                                        & = \left( x \otimes z\right) \oplus \left( y \otimes z\right)
			      \end{align}
			\item Multiplication by $\mathbf{0}$ annihilates $\R \times \R$:
			      \begin{align}
				      \mathbf{0} \otimes x & = \{a \circ b \,|\, a \in A, b \in \{\}\}                                                             \\
				                           & = \{\}                                    &  & \text{by definition of $\circ$, because no $b$ exists} \\
				                           & = x \otimes \mathbf{0}                                                                                \\
			      \end{align}
		\end{itemize}
		The Kleene start for $\mathcal{W}_{\text{lang}}$ given by:
		\begin{align}
			A^{\otimes n} & = \{ \bigotimes_{i=0}^{n} a_i \;|\; a_i \in A\}                                     \\
			A^*           & = \bigoplus_{n=0}^{\infty} A^{\otimes n}                                            \\
			              & = \bigoplus_{n=0}^{\infty} \left\{ \bigotimes_{i=0}^{n} a_i \;|\; a_i \in A\right\} \\
			              & = \left\{ \bigotimes_{i=0}^{n} a_i \;|\; a_i \in A, n \in \mathbb{Z} \right\}
		\end{align}
	\end{subquestion}
\end{question}
\begin{question}\\
	\begin{subquestion}
		Tropical semiring is $0$-closed:
		\begin{align}
			a \oplus \boldsymbol{0} & = \min(a, \boldsymbol{0})                                                 \\
			                        & = \min(a, 0)                                                              \\
			                        & = 0                       &  & \text{because $a \in \mathbb{R}_{\geq 0}$}
		\end{align}
		Arctic semiring is $0$-closed:
		\begin{align}
			a \oplus \boldsymbol{0} & = \max(a, \boldsymbol{0})                                                 \\
			                        & = \max(a, 0)                                                              \\
			                        & = 0                       &  & \text{because $a \in \mathbb{R}_{\leq 0}$}
		\end{align}
	\end{subquestion}
	\begin{subquestion}
		Proof by induction:\\
		Base case $i = 1$:
		\begin{align}
			M^1    & = M                             \\
			M_{ij} & = w_{ij} & \text{by def of $M$}
		\end{align}
		$w_{ij}$ is exactly the semiring-sum over all paths from $i$ to $j$ of length $1$. This holds because there is only one path of length $1$ from $i$ to $j$.\\
		Induction hypothesis:
		$M^{i}_{ij}$ is the semiring-sum over all paths from $i$ to $j$ of length $i$.\\
		Induction step $i \rightarrow i + 1$:\\
		\begin{align}
			M^{i + 1}    & = M^i \otimes M                                                                             \\
			M^{i+1}_{kj} & = \sum_{l=0}^{n} M^{i}_{kl} \otimes M_{lj} & \text{def. matrix mult.} \label{eq:mip1_entry}
		\end{align}
		In \cref*{eq:mip1_entry} we sum over the product of all possible paths of length $i$ from $k$ to another node $l$ and all possible paths of length $1$ from nodes $l$ to $j$. This sum is exactly the semiring-sum over all possible paths of length $i + 1$ from $k$ to $j$.\\
	\end{subquestion}
	\begin{subquestion}
		s
	\end{subquestion}
	\begin{subquestion}
		Per definition of the Kleene start, we have:
		\begin{align}
			M^* & = \bigoplus_{i=0}^{\infty} M^{\otimes i}
		\end{align}
		In b) we have shown that $M^{\otimes i}$ is the semiring-sum over all paths of length $i$ and in c) we have shown that the shortest path depends only on paths of length $l \leq N-1$. We also showed that under the $\bigoplus$ operation, only paths of length $l \leq N-1$ are considered. Therefore, we know that $M^*$ depends only on:
		\begin{align}
			M^{*} & = \bigoplus_{i=0}^{N-1} M^{\otimes i}
		\end{align}
	\end{subquestion}
	\begin{subquestion}
		We define a simple algorithm:\\
		\begin{algorithm}[H]
			\label{M_star}
			\caption{Matrix multiplication for Kleene star}
			$M$\;
			$M' \gets M$
			$M^* \gets \mathbf{0}$\;
			\For{$i = 0$ to $N-1$}{
			\For{$j = 0$ to len($M$)}{
			\For{$k = 0$ to len($M$)}{
			\For{$l = 0$ to len($M$)}{
			$M'_{j,k} \gets M'_{j,k} \oplus \left( M_{j,l} \otimes M_{l,k}\right)$
			}
			$M^*_{j,k} \gets M^*_{j,k} \oplus M'_{j,k}$
			}
			}
			$M \gets M'$\;
			}
		\end{algorithm}
		The inner for loops calculate the matrix multiplication, $M^{\otimes n}$, and the outer loop iterates $N-1$ times to calculate the Kleene star. Since each loop has $\mathcal{O}(N)$ iterations, the algorithm has a runtime of $\mathcal{O}(N^4)$.\\
	\end{subquestion}
\end{question}
\end{document}

%%% Local Variables:
%%% mode: latex
%%% TeX-master: t
%%% End: